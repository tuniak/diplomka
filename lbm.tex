\chapter{Lattice Boltzmann method}

In the previous chapter, we showed that Boltzmann equation is superior to the Navier-Stokes equations -- if we solve the Boltzmann equation, we can indirectly obtain the solution for the Navier-Stokes equations (NSE).

But since the distribution $f(\bm{x},\bm{u},t)$ is the function of seven variables, it is difficult equation to solve analytically.
Surprisingly, the numerical method for the Boltzmann equation turns out to be quiet simple -- both to implement and parallelise, and it results in the exact advection (as opposed to NSE solvers, that have major difficulty with the advection term). 

This numerical scheme evolved from the lattice-gas cellular automata that are the focus of our thesis, so we will explain it along the lines of LGCA, although explanation from various perspectives is possible.

% Hence, we will explain it along the lines of LGCA, although it can be explained on its own just fine.

%\section{Basics}
%The basic quantity used in lattice Boltzmann method is the velocity distribution $f_i(\bm{r},t)$, and it represents the density of particles with the lattice velocities $c_{i\alpha}$.  at the discrete times, and at the discrete points of the lattice.
%
%The lattice of the LBM is often denoted by symbol $DdQq$, where $d$ stands for the dimension, and $q$ is the number of lattice velocities, e.g. $D3Q19$ is the $3D$ lattice using $19$ lattice velocities, and is the most commonly employed lattice for $3D$.
\section{Basics}
The basic quantity for LBM is the velocity distribution $f_i(\bm{r},t)$.
Using the probability density function \ref{1pd}, it can be defined as
\begin{align*}
f_i(\bm{r},t) = f_1(\bm{r}, \bm{c_i}, t).
\end{align*}
and represents the density of the particles, but only at the discrete time steps and points of the lattice.

The LBM lattice is often denoted by the symbol $DdQq$, where $d$ is the dimension of the lattice, and $q$ is the number of lattice vectors, e.g. $D1Q3$, $D2Q9$ or $D3Q19$, which is the most common 3D variant with $19$ lattice vectors. Interestingly, it is the projection of the four dimensional FCHC lattice that we already described, see figure \ref{fchc}.

The mass and momentum density at the discrete point $(\bm{r},t)$ is given by
\begin{align} \label{rho}
\rho(\bm{r},t) = \sum_i f_i(\bm{r},t)
\end{align}
and
\begin{align} \label{dzej}
\bm{j}(\bm{r},t) = \sum_i c_i f_i(\bm{r},t),
\end{align}
which correspond to \ref{ddens} and \ref{mmom}, that we derived for the FHP model.

For isotermal lattice Boltzmann equation, the speed of sound is determined by the simple formula
\begin{align*}
p = c_s^2 \rho,
\end{align*}
and for the lattice sets we introduced above, it is
\begin{align*}
c_s^2 = \frac{1}{3} \frac{\Delta x^2}{\Delta t^2},
\end{align*}
It is a good custom to set the lattice spacing $\Delta x = 1$  and time step $\Delta t = 1$, and rescale the result to the physical units in demand.

The Boltzmann equation, discretized in space, time and velocity space \ref{Boltz} reads
\begin{align} \label{updt}
f_i(\bm{x} + \bm{c_i}\Delta t, t + \Delta t) = f_i(\bm{x},t) + \Omega_i(\bm{x},t),
\end{align}
which states that particles from the population $f_i(\bm{x},t)$ move to the point $\bm{x + c_i}$ at the next time step. Before that, particles are affected by the collision operator $\Omega$ that redistributes particles among the populations $f_i$.
Notice that this corresponds to the equation \ref{withcol} with one major difference -- in FHP the respective quantities were boolean (empty cell or particle inside), in LBM, these are continuous values of particle densities that collide and propagate.

The LBM version of the BGK collision operator \ref{bgk} is
\begin{align} \label{collar}
\Omega_i(f) = - \frac{f_i - f_i^{eq}}{\tau} \Delta t,
\end{align}
and gives us a good intuition about what collisions do -- the population numbers relax towards the equilibrium values at a rate given by $\tau$.
The equilibrium distribution is given by
\begin{align} \label{mbcomp}
f_i^{eq}(\bm{x},t) = w_i \rho \Big( 1 + \frac{\bm{u} \cdot \bm{c_i}}{c_s^2} + \frac{(\bm{u} \cdot \bm{c_i})^2}{2 c_s^4} - \frac{\bm{u} \cdot \bm{u}}{c_s^2}\Big),
\end{align}
where $w_i$ are the weights specific to the LBM lattice used, $\bm{u} = \frac{\bm{j}}{\rho}$ with $\rho$ and $\bm{j}$ given by \ref{rho} and \ref{dzej}.

\bigskip

In the previous chapter, we established the link between Boltzmann and Navier-Stokes equations, and derived the expressions for the shear viscosity \ref{shearv} and the stress tensor \ref{stresst}. Their discretized LBM variants are
\begin{align*}
\eta = c_s^2 \big( \tau - \frac{\Delta t}{2} \big)
\end{align*}
and
\begin{align*}
\Pi_{\alpha \beta} \approx - \big( 1 - \frac{\tau}{2 \tau} \big) \sum_i c_{i\alpha} c_{i\beta}(f_i - f_i^{eq}).
\end{align*}

\section{Collision and propagation}
Analogously to lattice-gas cellular automata, the update given by the Boltzmann equation \ref{updt} can be decomposed to
\begin{enumerate}
\item \textbf{Collision}
\begin{align}
f_i^*(\bm{x},t) = f_i(\bm{x},t) - \frac{\Delta t}{\tau}\big( f_i(\bm{x},t) - f_i^{eq}(\bm{x},t) \big)
\end{align}
where $f_i^*$ is the particle distribution after the collision.
\item \textbf{Propagation}
\begin{align*}
f_i(\bm{x} + \bm{c_i} \Delta t, t + \Delta t) = f_i^*(\bm{x},t).
\end{align*}
\end{enumerate}

Less formally, the update step can be described in the few points:
\begin{enumerate}
\item Use $f_i$ to compute $\rho$ and $\bm{j}$ via \ref{rho} and \ref{dzej}.
%\begin{align*}
%\begin{split}
%\rho = \sum_i f_i \\
%\bm{j} = \sum_i \bm{c_i} f_i
%\end{split}
%\end{align*}
\item Compute Maxwell-Boltzmann distribution $f_i^{eq}$ via \ref{mbcomp}.
\item Carry out the collision and the propagation.
\end{enumerate}

\section{Conclusion}
We are tempted to say that LBM is the the lattice-gas cellular automaton with continuous values of the cells and collision operator given by \ref{collar}.
Physically, however, this is a huge leap and they are not cellular automata in the true sense. The lattice-gas cellular automata are the microscopic toy models of the fluid, but due to their symmetries, realistic behavior emerges on the macroscopic level, while Lattice-Boltzmann method should be viewed as a numerical scheme for solving the Boltzmann equation. Thus it is beyond the scope of this thesis, and concludes our introduction to cellular automata.
%Thus, our story of cellular automata is in the end.