\chapter{Statistical description of FHP}

\section{From microcosmos to macroworld}

Although microdynamics of LGCA is physically unrealistic, laws of mass and momentum conservation are fulfilled in the nodes, as we showed by the end of last chapter.

By employing aparatus of statistical mechanics we will show that these microscopic conservation laws lead to physically realistic macroscopic description.

%If we insert 'cellular automaton' to the search on Arxiv,
%it floods us with thousand of papers and it loaded only most recent ones.
%It really became new kind of science, as the title of Wolfram's book promised 15 years ago.

\section{Liouville's theorem a.k.a. conservation of probabilities}
Let us define the phase space $\Gamma$ as the set of all possible states of the lattice $n(.)$.
Imagine we want to initialize cellular automaton with some macroscopic velocity $\bm{v_0}$, macroscopic pressure $p_0$, and macroscopic density $\rho_0$.
We can realize this macrostate by very many microstates of lattice $n(.)$.
We assign initial probability to each of these microstates: 
\begin{align}
P(0,s(.)) \geq 0.
\end{align}
Of course, probabilities over whole lattice are normalized so
$\sum_s(.) P(0,s(.)) = 1$.


In the statistical mechanics, Liouville's space state theorem postulates that the density of the phase space is constant. Microdynamics of our model implies equivalent theorem for LGCA.
\begin{align}
P(t+1, \mathcal{E} s(.)) = P(t, s(.))
\end{align}

It is obtained directly by applying the update formula
\begin{equation}
\mathcal{E} s(t,.) = s(t+1,.)
\end{equation}
However, for indeterministic model such as FHP, the conservation of probability is governed by the more general formula
\begin{equation}
P(t+1,\mathcal{S} n'(.)) = \sum_{n(.) \in \Gamma} \prod_{n(.)} A(n(\bm{r} \rightarrow n'(\bm{r})) P(t, n(.)),
\end{equation}
that constitutes the LGCA version of Chapman-Kolmogorov equation.

\section{Mean occupation numbers}
Motivated by the ensemble formalism of statistical physics, we define mean occupation numbers
\begin{equation}
N_i = \langle n_i \rangle = \sum_{s(.) \in \Gamma} n(s(.)) P(t,s(.)) .n
\end{equation}

This formula naturally suggests definition of the mean mass density
\begin{equation}
\rho(t,r) = \sum_i N_i(t,r)
\end{equation}
and the momentum density
\begin{equation}
j(t,r) = \sum_i c_i N_i
\end{equation}

Due to conservation of probabilities, the conservation laws \ref{cons1} implies conservation of the mean quantities
\begin{equation} \label{macro1}
\sum_i N_i(t+1,r+c_i) = \sum_i N_i(t,r) 
\end{equation}
\begin{equation} \label{macro2}
\sum_i c_i N_i(t+1,r+c_i) = \sum_i c_i N_i(t,r)
\end{equation}

\section{Equilibrium occupation number}
After laborious definitions in previous sections, we are ready for one of the central theorems of FHP model, stating that the equilibrium occupation numbers are given by Fermi-Dirac distribution. Its implications will haunt us until the last chapter.

We state it without the lengthy proof, but we recommend \cite{wolf} or \cite{frisch} for the non-believers.

\bigskip
Since the formalism that we were using in previous section is general, independent of the dimension of FHP, we could state the last theorem for the FHP-like automaton in arbitrary dimension D with b lattice vectors $c_i \in R^D,~i=1...b$.

\bigskip

\textbf{Theorem 1:}
The following statements are equivalent:
\begin{enumerate}
\item $N_i^{eq}$s are solutions of Chapman-Kolmogorov equation (reference)\\
\item $N_i^{eq}$s are solutions of set of b equations:
\begin{equation}
\Delta_i(N) = \sum_{nn'}(n'_i - n_i)A(n \rightarrow n')\prod_j N_j^{n_j}(1-N_j)^{1-n_j}
\end{equation} 

\item $N_i$ are given by Fermi-Dirac distribution
\begin{equation} \label{fd}
N_i^{eq} = \frac{1}{1 + \exp(h + \bm{q}.\bm{c_i})},
\end{equation}
where h is real number and \textbf{q} is D-dimensional vector.
\end{enumerate}

\bigskip

To express $N_i^{eq}$ explicitly as function of $\rho$ and $\bm{u}$, we employ technique of Lagrange multipliers with natural constraints
\begin{equation}
\rho = \sum_i N_i = \sum_i \frac{1}{1+ exp(h + \bm{q}.\bm{c_i})}
\end{equation}
\begin{equation}
\bm{u} = \sum_i \bm{c_i} \bm{N_i} = \sum_i \frac{\bm{c_i}}{1+ exp(h + \bm{q}.\bm{c_i})}
\end{equation}

Explicit solutions are available only in few special cases.
In general, we may use expansion for small Mach numbers ($\bm{u}/c_{\mathrm{sound}}$). By expansion up to second order, equilibrium distribution for $D$-dimensional FHP-like automaton reads \footnote{The expansion is required up to the second order, because the non-linear term in Navier-Stokes equations will emerge from the quadratic term}
\begin{equation} \label{eou}
N_i^{eq}(\rho,\bm{u}) = \frac{\rho}{b} + \frac{D\rho}{c^2 b}\bm{c_i}.\bm{u} = \rho G(\rho) Q_{i\alpha\beta}u_{\alpha}u_{\beta} + O(u^3)
\end{equation}
where 
\begin{align}
Q_{i\alpha\beta} = c_{i\alpha} c_{i\beta} - \frac{c^2}{D} \delta_{\alpha\beta}
\end{align}
and
\begin{align}
G(\rho) = \frac{D^2}{2c^4b}\frac{b-2\rho}{b-\rho}.
\end{align}

%Realizing that
%\begin{equation}
%\sum_i c_{i\alpha} c_{i\beta} = 
%\end{equation}

\section{Chapman-Enskog expansion}
Because relaxation towards equilibrium values happens in few updates of the automaton, it is standard procedure to expand occupation numbers $N_i(t,r)$ around equilibrium occupation numbers $N_i^{eq}(\rho,\bm{u})$:
\begin{equation} \label{chap}
N_i(t,r) = N_i^0(t,r) + \epsilon N_i^1(t,r) + \mathcal{O}(\epsilon^2) 
\end{equation} 
%Following formulas
%\begin{equation}
%\rho = \sum_i N_i(t,r) = \sum_i N_i^{eq}(\rho,\bm{u})
%\end{equation}
%\begin{equation}
%\bm{j} = \sum_i c_i N_i(t,r) = \sum_i c_i N_i^{eq}(\rho,\bm{u})
%\end{equation}
%imply that


The equations of mass and momentum conservation \ref{macro1} and \ref{macro2} can be equivalently stated in the form
\begin{equation} \label{macro_m}
\sum_i N_i(t+1,r+c_i) - N_i(t,r) = 0 ,
\end{equation}
\begin{equation} \label{macro_p}
\sum_i c_i (N_i(t+1,r+c_i) - N_i(t,r)) = 0,
\end{equation}
that is more suitable for our purpose.

Expansion of $N_i(t+1,r+c_i)$ around $N_i(t,r)$ leads to
\begin{equation} \label{rozvoj t+1}
\begin{split}
N_i(t+1,r+c_i) = N_i(t,r) + \partial_t N_i(t,r) + c_{i\alpha} \partial_{\alpha} N_i(t,r) \\ 
+ \frac{1}{2} \partial_t \partial_t N_i(t,r) + \frac{1}{2} c_{i\alpha}c_{i\beta} \partial_{\alpha} \partial_{\beta} N_i(t,r) + c_{i\alpha} \partial_t \partial_{\alpha} N_i(t,r) + \mathcal{O}(\partial^3).
\end{split}
\end{equation}
\bigskip
The expression above is the mixture of various physical phenomena -- diffusion, advection, propagation of sound waves or relaxation towards local equilibria. Each phenomena has its typical spatial and temporal scale, that corresponds to different powers of $\epsilon$, see the Table \ref{scalings}.

\begin{center} \label{scalings}
    \begin{tabular}{| l | l | l | l |}
    \hline
    \multicolumn{3}{|c|}{TEMPORAL SCALES}\\ \hline
    \textbf{Scale} & \textbf{Rescaling of time} & \textbf{Phenomena} \\ \hline
    1 step & t & Relaxation towards local equilibrium \\ \hline
    100 steps & $t_1 = \frac{1}{100} t = \epsilon t$ & advection, sound waves (perturbation of mass and density) \\ \hline
    10 000 steps & $t_2 = \frac{1}{10000} t = \epsilon^2 t$ & diffusion \\ \hline
    \end{tabular}
\end{center}


\begin{center}
    \begin{tabular}{| l | l | l | l |}
    \hline
    \multicolumn{3}{|c|}{SPATIAL SCALES}\\ \hline
    \textbf{Scale} & \textbf{Rescaling of length} & \textbf{Phenomena} \\ \hline
    1 lattice unit & \textbf{r} & Relaxation towards local equilibrium \\ \hline
    100 lattice units & $\bm{r_1} = \frac{1}{100} \bm{r} = \epsilon \bm{r}$ & diffusion, advection, sound waves\\ \hline
    \end{tabular}
\end{center}

To derive the hydrodynamical equation that we long for, we will exploit the multi-scale technique. It starts by grouping-up the terms of hydrodynamical temporal and spatial scales.


\bigskip
Using to rescaled the length and time, we deduce the rescaled differential operators
\begin{equation} \label{oper}
\begin{split}
\partial_t = \epsilon \partial_t^{(1)} + \epsilon^2 \partial_t^{(2)} \\
\partial_{\alpha} = \epsilon \partial_{\alpha}^{(1)}
\end{split}
\end{equation}

Now, we are ready to expand the conservation laws \ref{macro_m} and \ref{macro_p} by Chapman-Enskog. We insert expansion \ref{rozvoj t+1} of $N_i(t+1,r+c_i)$, then we insert expansion of $N_i(t,r)$ according to \ref{chap}. Finally, we substitute differential operators according to \ref{oper}.

We do not present the whole monstrous expansion, only the terms of order $\epsilon$, that we are interested in.

From conservation of mass \ref{macro_m} we get
\begin{equation}
\partial_t^{(1)} \sum_i N_i^{(0)} + \partial_{\beta} \sum_i c_{i\beta} N_i^{(0)} = 0,
\end{equation}
and from conservation of momentum \ref{macro_p} we get
\begin{equation}
\partial_t^{(1)} \sum_i c_{i\alpha} N_i^{(0)} + \partial_{\beta} \sum_i c_{i\alpha} c_{i\beta} N_i^{(0)} = 0.
\end{equation}

Using definition of mass density, and the following definition of the \textit{momentum flux tensor}
\begin{equation}
P^{(0)}_{\alpha\beta} := \sum_i c_{i\alpha} c_{i\beta} N_i^{(0)} = \sum_i c_{i\alpha} c_{i\beta} N_i^{eq}(\rho, \bm{u}).
\end{equation}
we can write the conservation laws in the shorter form
\begin{equation} 
\begin{split}
\partial_t^{(1)} \rho + \nabla^{(1)}(\rho \bm{u}) = 0, \\ 
\partial_t^{(1)} (\rho u_{\alpha}) + \nabla^{(1)} P^{(0)}_{\alpha\beta} = 0 \label{eul_primitive}
\end{split}
\end{equation}

In the first equation, we recognize the famous continuity equation, but we have some work to do with the second equation.

%Now is the time to observe, that:
%\begin{equation} \label{2mom}
%\sum_i c_{i\alpha}c_{i\beta} = 3\delta_{\alpha\beta}
%\end{equation}
%Using this observation and inserting formula for equilibrium occupation number \ref{eou} we can write:
%\begin{equation}
%P^{(0)}_{\alpha\beta} = \frac{\rho}{2} \delta_{\alpha\beta} 
%+ \rho G(\rho) \sum_i c_{i\alpha}c_{i\beta} Q_{i\gamma\delta} u_{\gamma} u_{\delta} + \mathcal{O}(u^4)
%\end{equation} 
Substituting \ref{eou} for $N_i^{eq}$ into momentum flux tensor, its components read (after some algebraic simplification)
\begin{equation} \label{FHPT}
\begin{split}
P_{xx}^{(0)} = \frac{\rho}{2}g(\rho)(u_x^2 - u_y^2) + \frac{\rho}{2},\\
P_{yy}^{(0)} = \frac{\rho}{2}g(\rho) (u_x^2 - u_y^2) + \frac{\rho}{2},\\
P_{xy}^{(0)} = P_{yx}^{(0)} = \rho g(\rho)u_xu_y,
\end{split}
\end{equation}
where we defined $g(\rho) =  \frac{3-\rho}{6 - \rho}$.

Unfortunately, it does not match the momentum flux tensor of Navier-Stokes equation
\begin{equation} \label{NST}
\begin{split}
P_{xx} = \rho \, u_x^2 + p\\
P_{yy} = \rho\, u_y^2 + p\\
P_{xy} = P_{yx} = \rho u_x u_y
\end{split}
\end{equation}

Let us examine why are these tensors different.

For low values of $\bm{u}^2$, pressure $p$ is given by isothermal relation \cite{wol} $p = \frac{\rho}{2} = \rho c_s^2$, where $c_s = \frac{1}{\sqrt{2}}$ is the speed of sound.

What about $g(\rho)$?  

The disease of FHP, FCHC and all lattice-gas cellular automata to follow is, that $g(\rho)$ is never equal to 1, as it is in Navier-Stokes equations.

The reason lies in the broken \textit{Galilei invariance} of LGCA,as they all have discrete rotational symmetry (by $60 \degree$ in FHP and between $60 \degree$ and $45 \degree$ in FCHC).

In the final chapter on theory of LGCA, we will show the fundamental treatment of this flaw. Symptomatic treatment, such as rescaling the time
\begin{align} \label{frac_resc}
t \rightarrow \frac{t}{g(\rho)},
\end{align}
does not solve all the associated problems (D'Humier et al. 1987).

However, using this rescaled time, and setting density to be constant ($\rho = \rho_0$ except for the pressure term \ref{press}), we get the familiar incompressible Euler equation
\begin{equation}
\frac{\partial \bm{u}}{\partial t} + (\bm{u} \bm{\nabla}) \bm{u} = -\bm{\nabla} P,
\end{equation}
where we used
\begin{equation} \label{press}
P = (\frac{\rho}{2\rho_0 g(\rho_0)} - \bm{u}^2).
\end{equation}.

This is as far as we can get in the first approximation.
The derivation of the Navier-Stokes equations
\begin{align}
\begin{split}
\bm{\nabla . u} &= 0, \\
\pd_t \bm{u} + (\bm{u \nabla})\bm{u} &= - \bm{\nabla} P + \nu \, \nabla^2 \bm{u}
\end{split}
\end{align}
would require terms of the order $\epsilon^2$  from the Chapman-Enskog expansion to add in the equations \ref{macro_m} and \ref{macro_p}, as done by Frisch \cite{frisch}.

%The second term is the advection term, and we may define \textit{momentum advection tensor} in first approximation:
%\begin{equation}
%T^{(MA)}_{\alpha\beta\gamma\delta} = \sum_i c_{i\alpha}c_{i\beta} Q_{i\gamma\delta} = \sum_i c_{i\alpha}c_{i\beta}(c_{i\gamma}c_{i\delta} - \frac{1}{2}\delta_{\alpha\beta}) = \frac{3}{4} (\delta_{\alpha\gamma} \delta_{\beta\delta} + \delta_{\alpha\delta} + \delta_{\beta\gamma} - \delta_{\alpha\beta}\delta_{\gamma\delta}
%\end{equation}


%Rozvoj $N_i$ okolo $N_i^{eq}$:\\
%$N_i^0 := N_i^{eq}$\\
%$N_i(t,\bm{r}) = N_i^0(r,t) + \epsilon N_i^1(t,r) + O(\epsilon^2)$
%these implies:\\
%$\sum_i N_i^{(1)} = 0$\\
%$\sum_i c_i N_i^{(1)} = 0$\\