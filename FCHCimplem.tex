\chapter{Implementation of FCHC}

In the chapter blabla, we inspected FCHC automaton theoretically, you should be familiar with it before we start.

According to \cite{Wolf}, (find idea on lack of implementation).
And field of practical applications is ruled by Lattice Boltzmann model nowadays, so we really do not find many applications.

As the more advanced "lattice Boltzmann model" evolved from LGCA, it attracted 
And nowadays, the "lattice Boltzmann model" has taken over the application.

Due to huge number of state of the node and huge number of colli
As state of the node is represented by 24 bits.
Therefore, integer with its 32 bits is enough to represent both node and obstacle (24 bits + 1 bit).

\bigskip

Not so long ago, table that would specify collision rules for $2^{24}$ states was problematic - it would require approximately $2^{24} \cross 10 \cross 4 bytes = 640$MB memory to manage.

Henon proposed algorithm for collision that would by-pass this obstacle, compute the possible set of states in reasonable time and chose optimal state.

However, the algorithm is quiet expensive on computation time, especially comparing to pair-interaction LGCA.

Therefore, we have chosen the following approach for resolving collisions:
\begin{enumerate}
\item At the beginning of the computation, before the simulation of the flow starts, we create table of $2^{24}$ entries. Integer index of the entry is the state of the node, and using Henon algorithm, we compute the optimal isometries for the state.
\item Collisions during simulation are resolved by choosing the resulting state from the table, instead of computing the optimal isometries again and again. We are not generating the random numbers to chose the resulting state, as it is expensive on resources and was significantly slowing the algorithm in the testing phase. However, we believe that choosing random states is not necessary and our solution for choosing the optimal isometries leads to the uniform distribution due to large number of nodes (around $10^{8}$).

\end{enumerate}

Excerpt of the source code relevant to the creation of the table and collision follows.

\section{Algorithm for the creation of the table}
\begin{lstlisting}
/* The node is represented as the integer, first 24 bits corresponds to the cells. If the value of the cell is 1, it is occupied by particle, otherwise it is 0. */
   
#define C1 1
#define C2 (C1<<1)
#define C3 (C1<<2)
#define C4 (C1<<3)
#define C5 (C1<<4)
#define C6 (C1<<5)
#define C7 (C1<<6)
#define C8 (C1<<7)
#define C9 (C1<<8)
#define C10 (C1<<9)
#define C11 (C1<<10)
#define C12 (C1<<11)
#define C13 (C1<<12)
#define C14 (C1<<13)
#define C15 (C1<<14)
#define C16 (C1<<15)
#define C17 (C1<<16)
#define C18 (C1<<17)
#define C19 (C1<<18)
#define C20 (C1<<19)
#define C21 (C1<<20)
#define C22 (C1<<21)
#define C23 (C1<<22)
#define C24 (C1<<23)

/* We assigned the 25th bit to the obstacle. If it is 1, this node is part of the obstacle and is not occupied by the particles (thus if this bit is 1, all other bits are 0) */
#define OBS (C1<<24)

/* We create the array of the cells, so we can efficiently iterate over the cells */
int C[24] = {
	C1,  C2,  C3,  C4,
	C5,  C6,  C7,  C8,
	C9,  C10, C11, C12,
	C13, C14, C15, C16,
	C17, C18, C19, C20,
	C21, C22, C23, C24
};


/* Array of the cells, such that Reverse[i] lies on the diagonal to the C[i] from above */
int Reverse[24] = {
	C4, C3, C2, C1,
	C8, C7, C6, C5,
	C12,C11,C10,C9,
	C16,C15,C14,C13,
	C20,C19,C18,C17,
	C24,C23,C22,C21
};


/* Lattice velocities. The lattice velocity c[i] corresponds to the cell C[i] */
const int c[24][4] = {
	{ 1,1,0,0 },
	{ 1,-1,0,0 },
	{ -1,1,0,0 },
	{ -1,-1,0,0 },
	
	{ 1,0,1,0 },
	{ 1,0,-1,0 },
	{ -1,0,1,0 },
	{ -1,0,-1,0 },

	{ 1,0,0,1 },
	{ 1,0,0,-1 },
	{ -1,0,0,1 },
	{ -1,0,0,-1 },

	{ 0,1,1,0 },
	{ 0,1,-1,0 },
	{ 0,-1,1,0 },
	{ 0,-1,-1,0 },

	{ 0,1,0,1 },
	{ 0,1,0,-1 },
	{ 0,-1,0,1 },
	{ 0,-1,0,-1 },

	{ 0,0,1,1 },
	{ 0,0,1,-1 },
	{ 0,0,-1,1 },
	{ 0,0,-1,-1 }
};

/* This function swap the i-th and j-th bit of the node n */
int switchBits(int n, int i, int j)
{
	int a, b;
	a = n & C[i];
	b = n & C[j];
	if (a > 0 && b == 0)
	{
		n ^= C[i];
		n |= C[j];
	}
	else if (b > 0 && a == 0)
	{
		n |= C[i];
		n ^= C[j];
	}
	return n;
}



/* Implementation of the isometries \Sigma_1 and \Sigma_2 on the node n*/
/* As all isometries that will follow, it is achieved by swapping the appropriate bits in the node */
/* It also transforms the momentum vector q */
/* The parameter j is dummy parameter, we need it so that all isometries have the same set of parameters */
int sigma(int n, int* q, int i, int j)
{
	int a = q[0];
	int b = q[1];
	int c = q[2];
	int d = q[3];

	if (i == 1)
	{
		q[0] = ( a + b + c - d ) / 2;
		q[1] = ( a + b - c + d ) / 2;
		q[2] = ( a - b + c + d ) / 2;
		q[3] = (-a + b + c + d ) / 2;

		n = switchBits(n, 1, 21);
		n = switchBits(n, 2, 22);
		n = switchBits(n, 5, 17);
		n = switchBits(n, 6, 18);
		n = switchBits(n, 8, 12);
		n = switchBits(n, 11, 15);
	}
	else
	{
		q[0] = ( a + b + c + d ) / 2;
		q[1] = ( a + b - c - d ) / 2;
		q[2] = ( a - b + c - d ) / 2;
		q[3] = ( a - b - c + d ) / 2;

		n = switchBits(n, 1, 20);
		n = switchBits(n, 2, 23);
		n = switchBits(n, 5, 16);
		n = switchBits(n, 6, 19);
		n = switchBits(n, 9, 12);
		n = switchBits(n, 10, 15);
	}
	return n;
}

/* Isometry S_i is the reflection over plane x_i */
int S(int n, int* q, int i, int j)
{
	switch (i)
	{
	case 1:
		q[0] = -q[0];
		n = switchBits(n, 0, 2);
		n = switchBits(n, 1, 3);
		n = switchBits(n, 4, 6);
		n = switchBits(n, 5, 7);
		n = switchBits(n, 8, 10);
		n = switchBits(n, 9, 11);
		break;
	case 2:
		q[1] = -q[1];
		n = switchBits(n, 0, 1);
		n = switchBits(n, 2, 3);
		n = switchBits(n, 12, 14);
		n = switchBits(n, 13, 15);
		n = switchBits(n, 16, 18);
		n = switchBits(n, 17, 19);
		break;
	case 3:
		q[2] = -q[2];
		n = switchBits(n, 4, 5);
		n = switchBits(n, 6, 7);
		n = switchBits(n, 12, 13);
		n = switchBits(n, 14, 15);
		n = switchBits(n, 20, 22);
		n = switchBits(n, 21, 23);
		break;
	case 4:
		q[3] = -q[3];
		n = switchBits(n, 8, 9);
		n = switchBits(n, 10, 11);
		n = switchBits(n, 16, 17);
		n = switchBits(n, 18, 19);
		n = switchBits(n, 20, 21);
		n = switchBits(n, 22, 23);
		break;
	default:
		break;
	}
	return n;
}

/* Isometry P_ij, reflecton over the plain x_i = x_j */
int P(int n, int*q, int i, int j)
{
	int a = i - 1;
	int b = j - 1;
	int qa = q[a];
	q[a] = q[b];
	q[b] = qa;

	if (i == 1)
	{
		if (j == 2)
		{
			n = switchBits(n, 1, 2);
			n = switchBits(n, 4, 12);
			n = switchBits(n, 5, 13);
			n = switchBits(n, 6, 14);
			n = switchBits(n, 7, 15);
			n = switchBits(n, 8, 16);
			n = switchBits(n, 9, 17);
			n = switchBits(n, 10, 18);
			n = switchBits(n, 11, 19);
		}
		else if (j == 3)
		{
			n = switchBits(n, 0, 12);
			n = switchBits(n, 1, 14);
			n = switchBits(n, 2, 13);
			n = switchBits(n, 3, 15);
			n = switchBits(n, 5, 6);
			n = switchBits(n, 8, 20);
			n = switchBits(n, 9, 21);
			n = switchBits(n, 10, 22);
			n = switchBits(n, 11, 23);
		}
		else if (j == 4)
		{
			n = switchBits(n, 0, 16);
			n = switchBits(n, 1, 18);
			n = switchBits(n, 2, 17);
			n = switchBits(n, 3, 19);
			n = switchBits(n, 4, 20);
			n = switchBits(n, 5, 22);
			n = switchBits(n, 6, 21);
			n = switchBits(n, 7, 23);
			n = switchBits(n, 9, 10);
		}
	}
	else if (i == 2)
	{
		if (j == 3)
		{
			n = switchBits(n, 0, 4);
			n = switchBits(n, 1, 5);
			n = switchBits(n, 2, 6);
			n = switchBits(n, 3, 7);
			n = switchBits(n, 13, 14);
			n = switchBits(n, 16, 20);
			n = switchBits(n, 17, 21);
			n = switchBits(n, 18, 22);
			n = switchBits(n, 19, 23);
		}
		else if (j == 4)
		{
			n = switchBits(n, 0, 8);
			n = switchBits(n, 1, 9);
			n = switchBits(n, 2, 10);
			n = switchBits(n, 3, 11);
			n = switchBits(n, 12, 20);
			n = switchBits(n, 13, 22);
			n = switchBits(n, 14, 21);
			n = switchBits(n, 15, 23);
			n = switchBits(n, 17, 18);
		}
	}
	else if (i == 3 && j == 4)
	{
		n = switchBits(n, 4, 8);
		n = switchBits(n, 5, 9);
		n = switchBits(n, 6, 10);
		n = switchBits(n, 7, 11);
		n = switchBits(n, 12, 16);
		n = switchBits(n, 13, 17);
		n = switchBits(n, 14, 18);
		n = switchBits(n, 15, 19);
		n = switchBits(n, 21, 22);
	}
	return n;
}

/* Computes the total momentum of the node */
int* momenta(int node)
{
	int*q = new int[4]{ 0,0,0,0 };

	for (int i = 0; i < 24; ++i)
	{
		/* if the cell C[i] is occupied by particle, count its momentum */
		if (C[i] & node)
		{
			for (int j = 0; j < 4; ++j)
			{
				q[j] += c[i][j];
			}
		}
	}
	return q;
}

/* We need to remember all the isometries that normalize the momentum, so we can transform it back to its original momentum */
/* We define this array so that we can reffer to the isometry by the single index */
int(*iso[])(int n, int*q, int i, int j) = { S,P,sigma };


/* This function does all isometries that normalized the node, but in the reverse order, so that original momentum is achieved */
void goBack(int**steps, int* nodes, int*q, int step, int length)
{
	for (int i = 1; i < length; i++)
	{
		for (int j = step - 1; j >= 0; --j)
		{
			nodes[i] = iso[steps[j][0]](nodes[i], q, steps[j][1], steps[j][2]);
		}
	}
}

/* Creates the entry for the state 'n' that we will insert into the table of collisions */
/* It is implementation of the Henon algorithm, that computes the set of optimal isometries */ 
int* newNode(int n, int**steps)
{
	int*q = momenta(n);

	int step = 0;
	
	// invert negative q[i]
	for (int i = 0; i < 4; i++)
	{
		if (q[i] < 0)
		{
			//S changes sign of q[i];
			n = S(n, q, i+1, 0);
			//we record each step to the array "steps" so we can go back
			steps[step][0] = 0;
			steps[step][1] = i+1;
			steps[step][2] = 0;
			++step;
		}
	}
	// sort q[i] from high to low
	int highIndex;
	int highValue;
	for (int i = 0; i < 4; ++i)
	{
		highIndex = i;
		highValue = q[i];
		for (int j = i+1; j < 4; ++j)
		{
			if (q[j] > highValue)
			{
				highValue = q[j];
				highIndex = j;
			}
		}
		if (highIndex > i)
		{
			n = P(n, q, i + 1, highIndex + 1);
			steps[step][0] = 1;
			steps[step][1] = i + 1;
			steps[step][2] = highIndex + 1;
			++step;
		}
	}
	// to fulfill second condition:
	if (q[3] > 0)
	{
		if (q[0] + q[3] == q[1] + q[2])
		{
			n = sigma(n, q, 2, 0);
			steps[step][0] = 2;
			steps[step][1] = 2;
			steps[step][2] = 0;
			++step;
		}
		else if (q[0] + q[3] > q[1] + q[2])
		{
			n = sigma(n, q, 1, 0);
			steps[step][0] = 2;
			steps[step][1] = 1;
			steps[step][2] = 0;
			++step;
		}
	}
	if (q[3] < 0)
	{
		n = S(n, q, 4, 0);
		//we record each step to the steps field, since we will go back
		steps[step][0] = 0;
		steps[step][1] = 4;
		steps[step][2] = 0;
		++step;
	}


	int* nodes;
	//class 12
	if (q[0] == 0)
	{
		int length = 13;
		nodes = new int[length];
		nodes[0] = length - 1;
		//S3 S1 P34 P12,  S4 S1 P34 P12,  S3 S2 P34 P12, S4 S2 P34 P12,  
		//S2 S1 P24 P13,  S4 S1 P24 P13,  S3 S2 P24 P13, S4 S3 P24 P13,
		//S2 S1 P23 P14,  S3 S1 P23 P14,  S4 S2 P23 P14, S4 S3 P23 P14

		nodes[1] = P(n, q, 1, 2);
		nodes[1] = P(nodes[1], q, 3, 4);
		nodes[1] = S(nodes[1], q, 1, 0);
		nodes[1] = S(nodes[1], q, 3, 0);

		nodes[2] = P(n, q, 1, 2);
		nodes[2] = P(nodes[2], q, 3, 4);
		nodes[2] = S(nodes[2], q, 1, 0);
		nodes[2] = S(nodes[2], q, 4, 0);

		nodes[3] = P(n, q, 1, 2);
		nodes[3] = P(nodes[3], q, 3, 4);
		nodes[3] = S(nodes[3], q, 2, 0);
		nodes[3] = S(nodes[3], q, 3, 0);

		nodes[4] = P(n, q, 1, 2);
		nodes[4] = P(nodes[4], q, 3, 4);
		nodes[4] = S(nodes[4], q, 2, 0);
		nodes[4] = S(nodes[4], q, 4, 0);

		nodes[5] = P(n, q, 1, 3);
		nodes[5] = P(nodes[5], q, 2, 4);
		nodes[5] = S(nodes[5], q, 1, 0);
		nodes[5] = S(nodes[5], q, 2, 0);

		nodes[6] = P(n, q, 1, 3);
		nodes[6] = P(nodes[6], q, 2, 4);
		nodes[6] = S(nodes[6], q, 1, 0);
		nodes[6] = S(nodes[6], q, 4, 0);

		nodes[7] = P(n, q, 1, 3);
		nodes[7] = P(nodes[7], q, 2, 4);
		nodes[7] = S(nodes[7], q, 2, 0);
		nodes[7] = S(nodes[7], q, 3, 0);

		nodes[8] = P(n, q, 1, 3);
		nodes[8] = P(nodes[8], q, 2, 4);
		nodes[8] = S(nodes[8], q, 3, 0);
		nodes[8] = S(nodes[8], q, 4, 0);

		nodes[9] = P(n, q, 1, 4);
		nodes[9] = P(nodes[9], q, 2, 3);
		nodes[9] = S(nodes[9], q, 1, 0);
		nodes[9] = S(nodes[9], q, 2, 0);
		
		nodes[10] = P(n, q, 1, 4);
		nodes[10] = P(nodes[10], q, 2, 3);
		nodes[10] = S(nodes[10], q, 1, 0);
		nodes[10] = S(nodes[10], q, 3, 0);

		nodes[11] = P(n, q, 1, 4);
		nodes[11] = P(nodes[11], q, 2, 3);
		nodes[11] = S(nodes[11], q, 2, 0);
		nodes[11] = S(nodes[11], q, 4, 0);

		nodes[12] = P(n, q, 1, 4);
		nodes[12] = P(nodes[12], q, 2, 3);
		nodes[12] = S(nodes[12], q, 3, 0);
		nodes[12] = S(nodes[12], q, 4, 0);

		goBack(steps, nodes, q, step, length);

	}
	//class 11
	else if (q[1] == 0)
	{
		int length = 7;
		nodes = new int[length];
		nodes[0] = length - 1;
		//S4 S2 P23,   S4 S3 P23,   S3 S2 P24,   
		//S4 S3 P24,   S3 S2 P34,   S4 S2 P34

		nodes[1] = P(n, q, 2, 3);
		nodes[1] = S(nodes[1], q, 2, 0);
		nodes[1] = S(nodes[1], q, 4, 0);

		nodes[2] = P(n, q, 2, 3);
		nodes[2] = S(nodes[2], q, 3, 0);
		nodes[2] = S(nodes[2], q, 4, 0);

		nodes[3] = P(n, q, 2, 4);
		nodes[3] = S(nodes[3], q, 2, 0);
		nodes[3] = S(nodes[3], q, 3, 0);

		nodes[4] = P(n, q, 2, 4);
		nodes[4] = S(nodes[4], q, 3, 0);
		nodes[4] = S(nodes[4], q, 4, 0);

		nodes[5] = P(n, q, 3, 4);
		nodes[5] = S(nodes[5], q, 2, 0);
		nodes[5] = S(nodes[5], q, 3, 0);

		nodes[6] = P(n, q, 3, 4);
		nodes[6] = S(nodes[6], q, 2, 0);
		nodes[6] = S(nodes[6], q, 4, 0);
		

		goBack(steps, nodes, q, step, length);

	}
	//class 10
	else if (q[2] == 0 && q[0]==q[1])
	{
		int length = 7;
		nodes = new int[length];
		nodes[0] = length - 1;
		//S3,P34,P12,   S4,P34,P12,   S4,S3,sigma1,   S4,S3,P34,P12,sigma1
		//S4,S3,sigma2,   P34,P12,sigma2
		nodes[1] = P(n, q, 1, 2);
		nodes[1] = P(nodes[1], q, 3, 4);
		nodes[1] = S(nodes[1], q, 3, 0);

		nodes[2] = P(n, q, 1, 2);
		nodes[2] = P(nodes[2], q, 3, 4);
		nodes[2] = S(nodes[2], q, 4, 0);

		nodes[3] = sigma(n, q, 1, 0);
		nodes[3] = S(nodes[3], q, 3, 0);
		nodes[3] = S(nodes[3], q, 4, 0);

		nodes[4] = sigma(n, q, 1, 0);
		nodes[4] = P(nodes[4], q, 1, 2);
		nodes[4] = P(nodes[4], q, 3, 4);
		nodes[4] = S(nodes[4], q, 3, 0);
		nodes[4] = S(nodes[4], q, 4, 0);

		nodes[5] = sigma(n, q, 2, 0);
		nodes[5] = S(nodes[5], q, 3, 0);
		nodes[5] = S(nodes[5], q, 4, 0);

		nodes[6] = sigma(n, q, 2, 0);
		nodes[6] = P(nodes[6], q, 1, 2);
		nodes[6] = P(nodes[6], q, 3, 4);

		goBack(steps, nodes, q, step, length);
	}
	//class 9
	else if (q[2] == 0 && q[0] > q[1])
	{
		int length = 4;
		nodes = new int[length];
		nodes[0] = length - 1;
		
		nodes[1] = S(n, q, 3, 0);
		nodes[1] = S(nodes[1], q, 4, 0);
		
		nodes[2] = P(n, q, 3, 4);
		nodes[2] = S(nodes[2], q, 3, 0);
		
		nodes[3] = P(n, q, 3, 4);
		nodes[3] = S(nodes[3], q, 4, 0);

		goBack(steps, nodes, q, step, length);
	}
	//class 8
	else if (q[0]==q[1] && q[1] == q[2] && q[2] > q[3] && q[3]==0)
	{
		int length = 5;
		nodes = new int[length];
		nodes[0] = length - 1;
		//1
		nodes[1] = P(n, q, 1, 2);
		nodes[1] = P(nodes[1], q, 2, 3);
		//2
		nodes[2] = P(n, q, 1, 3);
		nodes[2] = P(nodes[2], q, 2, 3);
		//3
		nodes[3] = P(n, q, 1, 2);
		nodes[3] = P(nodes[3], q, 2, 3);
		nodes[3] = S(nodes[3], q, 4, 0);

		//4
		nodes[4] = P(n, q, 1, 3);
		nodes[4] = P(nodes[4], q, 2, 3);
		nodes[4] = S(nodes[4], q, 4, 0);

		goBack(steps, nodes, q, step, length);
	}
	//class 6 or 7
	else if (q[0]>q[1] && q[1]==q[2] && q[2] > q[3] && q[3] == 0)
	{
		//class 6
		if (q[0] == 2*q[1])
		{
			int length = 5;
			nodes = new int[length];

			nodes[0] = length - 1;

			nodes[1] = sigma(n, q, 1, 0);
			nodes[1] = S(nodes[1], q, 4, 0);

			nodes[2] = sigma(n, q, 2, 0);
			nodes[2] = S(nodes[2], q, 4, 0);

			nodes[3] = sigma(n, q, 1, 0);
			nodes[3] = P(nodes[3], q, 2, 3);
			nodes[3] = S(nodes[3], q, 4, 0);

			nodes[4] = sigma(n, q, 2, 0);
			nodes[4] = P(nodes[4], q, 2, 3);
			nodes[4] = S(nodes[4], q, 4, 0);

			goBack(steps, nodes, q, step, length);

		}
		//class 7
		else
		{
			int length = 2;
			nodes = new int[length];

			nodes[0] = length - 1;

			nodes[1] = P(n, q, 2, 3);
			nodes[1] = S(nodes[1], q, 4, 0);

			goBack(steps, nodes, q, step, length);
		}
		
	}
	//class 5
	else if (q[0]==q[1] && q[1]>q[2] && q[2] > q[3] && q[4] == 0)
	{
		int length = 2;
		nodes = new int[length];

		nodes[0] = length - 1;
		//1
		nodes[1] = P(n, q, 1, 2);
		nodes[1] = S(nodes[1], q, 4, 0);

		goBack(steps, nodes, q, step, length);
	}
	//class 3,4
	else if (q[0]>q[1] && q[1]>q[2] && q[2] > q[3] && q[3] == 0)
	{
		//class 3
		if (q[0]==q[1]+q[2])
		{
			int length = 3;
			nodes = new int[length];
			
			nodes[0] = length - 1;

			nodes[1] = sigma(n, q, 1, 0);
			nodes[1] = S(nodes[1], q, 4, 0);

			nodes[2] = sigma(n, q, 2, 0);
			nodes[2] = S(nodes[2], q, 4, 0);

			goBack(steps, nodes, q, step, length);

		}
		//class 4
		else
		{
			int length = 2;
			nodes = new int[length];

			nodes[0] = length - 1;

			nodes[1] = S(n, q, 4, 0);

			goBack(steps, nodes, q, step, length);
		}
	}
	//class 2
	else if (q[0]==q[1] && q[1]==q[2] && q[2]>q[3] && q[3]>0)
	{
		int length = 3;
		nodes = new int[length];

		nodes[0] = length - 1;

		nodes[1] = P(n, q, 1, 2);
		nodes[1] = P(nodes[1], q, 2, 3);

		nodes[2] = P(n, q, 1, 3);
		nodes[2] = P(nodes[2], q, 2, 3);

		goBack(steps, nodes, q, step, length);
	}
	//class 1
	else if (q[0]==q[1] && q[1]>q[2] && q[2] > q[3] && q[3] > 0)
	{
		int length = 2;
		nodes = new int[length];

		nodes[0] = length - 1;

		nodes[1] = P(n, q, 1, 2);

		goBack(steps, nodes, q, step, length);
	}
	else
		nodes = nullptr;
	return nodes;
}

/*  This is the high-level function that creates the table of isometries */
/* For all 2^24 possible states of the node, it calls the function newNode and save the set of new states */
void fillTable(int** table)
{
	int n;
	
	int ** steps = new int*[20];
	for(int i = 0; i<20; ++i)
	   steps[i] = new int[3];	

	for ( n = 0; n < OBS; ++n)
		table[n] = newNode(n, steps);
}
\end{lstlisting}

\section{Algorithm for collision}
Once we have the table specifying optimal isometrical states ready (by the function \textit{fillTable} above),
the algorithm for collision is to implement. The only non-trivial task is to achieve uniform distribution of chosen isometries without generation of the random numbers, as it is expensive on CPU time.

\begin{lstlisting}
/* For all nodes 'n' on the grid, this function looks in the table and choose from optimal final states */

void Collision(int***grid, int**table, int t, int X, int Y, int Z)
{
	int x,y,z;

	int n;
	int k;
#pragma omp parallel for private (x,y,z,n,k)
	for (x = 0; x < X; ++x)
		for (y = 0; y < Y; ++y)
			for (z = 0; z < Z; ++z)
			{
				//We find that node with position [x][y][y] on the grid is in the state n 
				n = grid[x][y][z];
				// If there is obstacle in the node, we skip it.
				if (n & OBS)
					break;
				/* k is initialized by the external parameter t,
				    and k is moduled by number of optimal isometries for the state n */
				k = (t % table[n][0]) + 1;
				
				// we assign the k-th optimal state to the node 
				grid[x][y][z] = table[n][k];
				
				//every step, we increment the external parameter, so we hopefully achieve the uniform distribution of k (there is huge number of nodes on the grid)
				++t;
			}
}
\end{lstlisting}

\section{Propagation in FCHC}
Although the collisions are resolved in the four dimensional space (and the particles are the four dimensional objects), for the purpose of propagation, we can simply 'forget' its four-dimensional nature.

The nodes and particles are sitting on the three dimensional grid and propagate along the 3D projection of lattice vectors \ref{fchc}.

\begin{lstlisting}
/* The propagation is happening in all the nodes at once, but in the computer simulation, it is happening sequentially, so we need to use to grids (propagation happening on the one grid would overwrite some nodes by new particles before the old particles propagated away). */
/* Hence, we are using the two grids, 'int***even' and 'int***odd'. 'Even' grid is occupied at even times, 'Odd' grid is occupied at odd times. For example, in odd time, first parameter of Propagation (int***from) will be 'Odd' and second parameter (int***to) will be 'Even' */

void Propagation(int***from, int***to, int**table, int X, int Y, int Z)
{
	int x,y,z;
	int i;
	int n;
	int new_x, new_y, new_z;

#pragma omp parallel for private (n, new_x, new_y, new_z, x, y, z, i)
	for ( x = 0; x < X; ++x)
		for ( y = 0; y < Y; ++y)
			for ( z = 0; z < Z; ++z)
			{
				// The current state of the node at [x][y][z]				
				n = from[x][y][z];
				// We check every cell
				for ( i = 0; i < 24; ++i)
				{
					// If the cell C[i] is occupied by particle, we propagate it to the corresponding node. 
					if (n & C[i])
					{
						new_x = PeriodicBC(x + c[i][0], X);
						new_y = PeriodicBC(y + c[i][1], Y);
						new_z = PeriodicBC(z + c[i][2], Z);
						// If there is obstacle in the node, particle's velocity is reversed.
						// e.g. particle with momentum [1,0,-1,0] gains momentum [-1,0,1,0] by the reflection from the obstacle
						if (to[new_x][new_y][new_z] & OBS)
							to[new_x][new_y][new_z] |= Reverse[i];
						else
							to[new_x][new_y][new_z] |= C[i];
					}
				}
				from[x][y][z] = 0;
			}
}
\end{lstlisting}
