\chapter{Probabilistic tools in turbulence}
In this chapter, we will remind basic concepts of probability and statistics, and introduce more advanced tools, that are frequently used in probabilistic description of turbulence. 


%So far, the most successful approach in turbulence research is by the means of statistical analysis.
%
%And life-long research of Frisch et. al., who conceived LGCA.
%
%As we saw the previous chapters, it wanted statistical physics to explain how LGCA works.
%
%LGCA and Turbulence are such a nice couple.
%
%Our thesis thrives to be a little .
%So far, the most successful approach in turbulence research is statistical analysis.
%
%\bigskip
%
%In this chapter, we introduce some basic and more advanced concepts of probability that will be handy in our research. 
%
%Following text requires prior knowledge of some probability and statistics. Basic concepts will be slightly reminded and will explain more interesting and advanced stuff in detail.
\section{Random variable}

Let us define random variable as a map
\begin{align}
v: \Omega \rightarrow \mathbb{R}
\end{align}
from the probability space to the real numbers.

A good example of random variable is the x-component of a velocity at the specific point, in the specific time. This is example of continuous random variable. In the whole section, we will consider only continuous variables and functions, as all the physical quantities of our interest are continuous.

\bigskip
To define the central concept of \textit{probability density function}.

It is useful to define a cumulative probability function
\begin{align}
F(x) = \mathrm{Prob}\left\{v(\omega) < x\right\}
\end{align}
which describes the probability that random variable $v$ takes a value smaller than $x$. Since probability is non-negative measure, $F(x)$ is non-decreasing function. 

The famous \textit{probability density
Hence, its derivative is a non-negative function,
\begin{align}
p(x) = \frac{\dd F(x)}{\dd x} \geq 0.
\end{align}

The famous 
And it happens to be the famous $probability~density~function$

Loosely speaking, $p(x)dx$ is the probability that random variable takes value between $x$ and $x+dx$.

Of course, probability is normalized to one:
\begin{align}
\int_{\mathbb{R}} p(x) \, dx = 1.
\end{align}

Using $probability~density~function$, we can express mean value of $v$
\begin{align}
\langle v \rangle = \int_{\mathbb{R}} x \, p(x) \, dx ,
\end{align}
variance of $v$
\begin{align}
\sigma^2 = \langle v^2 \rangle = \int_{\mathbb{R}} x^2 \, p(x) \, dx ,
\end{align}
or any other statistical moment of $v$
\begin{align}
\langle v^m \rangle = \int_{\mathbb{R}} x^m \, p(x) \, dx.
\end{align}

The centered
In the following text, we suppose that
\begin{align*}
\langle v \rangle = 0,
\end{align*}
so the statistical moments are actually central statistical moments
\begin{align*}
\langle v^m_{central} \rangle = \langle v^m - \langle v \rangle \rangle = \langle v^m \rangle
\end{align*}

For non-centered variables $w$, all statements will be valid for $w' = w - \langle w \rangle$.

\section{Characteristic function}

Of special interest is the Fourier transform of $probability~density~function$
\begin{align*}
K(z) = \int_{\mathbb{R}} e^{izv} p(x) dx = \langle e^{izv} \rangle.
\end{align*}
It is called $characteristic~function$ of variable $v$.

It is interesting for couple of reasons. For example, if we consider sum of two random variables $u+v$, corresponding $p.d.f.$ would be convolution of their respective $p.d.f.s$. But its characteristic function is product of respective characteristic functions. Hooray!

\bigskip

But for Guassian random variables, characteristic function leads us to various useful formulas.

\section{Gaussian random variables}
Gaussian random variable can be characterized by - the characteristic function.
Because its characteristic function can be simplified to
\begin{align*}
K(z) = \langle e^{izv} \rangle = e^{-\frac{1}{2} \sigma^2 \, v^2}
\end{align*}


\section{Vector random variable}
Let us go back to the example of random variable - the x-coordinate of velocity at specific time $t$ and specific position $\bm{r}$ in turbulent flow $v_x$.

Not surprisingly, vector random variable would be velocity vector $\bm{v} = (v_x,v_y,v_z)$ in turbulent flow.

Some definitions from previous sections are easy to generalize to vector variables and we omit them, but sometimes we need to redefine them.

In general, moments of the vector random variables are tensors
\begin{align}
\langle v_{i_1}, v_{i_2}, ..., v_{i_n} \rangle
\end{align}

The most important among them is the covariance
\begin{align}
\Gamma_{ij} = \langle v_i v_j \rangle
\end{align}

\bigskip
Another definition worth of commentary is the definition of Gaussian random variables.
Vector random variable $\bm{v}$ is Gaussian, if scalar product $\bm{v}.\bm{c}$ is Gaussian (in the sense of previous definition) for every $\bm{c} \in R^n$.

And if $\bm{v}$ is Gaussian, its characteristic function takes simple form
\begin{align}
K(\bm{z}) = \langle e^{i\bm{z}.\bm{v}} \rangle = e^{-\frac{1}{2} z_i z_j \Gamma_{ij}}
\end{align}

We see that characteristic function $K(\bm{z})$ depends only on the covariance tensor $\Gamma_{ij}$.

\bigskip


Starting from that, we could show that any statistical moment of $\bm{v}$ can be expressed as function of covariance. For example
\begin{align}
\langle v_i v_j v_k v_l \rangle = \Gamma_{ij}\Gamma_{kl} + \Gamma_{il}\Gamma_{jk} + \Gamma_{ik}\Gamma_{jl}.
\end{align}

We state it without proof, as it would require too much paper.

\section{Random function}
Random function can be defined as "the family of random variables" - sequence that is indexed by continuous time and spatial variables (vector random variable is sequence of random variables indexed by enumerable and finite set, e.g. $I=\big\{1,2,..N\big\}$.

Going back to our velocity example, random function would be the whole velocity field in turbulent flow
$\bm{v}(t,\bm{r})$.

\bigskip

Definitions from the previous sections are easy to upgrade from random functions.
Just bear in mind, that scalar product of random functions $u$ and $v$ means
\begin{align}
\int_{\Omega} u \, v \, dx.
\end{align}
This definition of scalar product is correct under assumption that random functions are real and square-integrable (from $L^2$ space).


\section{Stationary random variable}
Loosely said, $v(t)$ is stationary if $v(t)$ and $v(t+h)$ have same statistical properties.

For example, for correlation function we have
\begin{align}
\langle v(t_1+h) v(t_2 + h) \rangle = \langle v(t_1) v(t_2) \rangle = \Gamma(t_1 - t_2).
\end{align}

Hence, variance $\langle v(t)^2 \rangle = \Gamma(0)$ does not depend on time at all.
But be careful, difference of stationary function is not necessarily stationary. 
For example, if $v(t)$ is stationary, then $v(2t)$ is also stationary, because its variance

%\begin{align}
%\langle v(2t)^2 \rangle = \Gamma(2t - 2t) = \Gamma(0)
%\end{align}
does not depend on time.
However, their difference
\begin{align}
\langle (v(2t) - v(t))^2 \rangle = \langle v^2(t) \rangle + \langle v^2(2t) \rangle + 2\langle v(t)\,v(2t) \rangle = 2\Gamma(0) + \Gamma(2t - t)
\end{align} 
does depend on time.

\section{Spectrum of stationary random functions}

To analyze stationary random function, we have very powerful tool at our reach - the expansion of a random variable into Fourier harmonics:
\begin{align}
v(t,\omega) = \int_{R} e^{itf} \hat{v}(f,\omega) df
\end{align}
where $f$ denotes frequency.

To filter-out temporal scales greater than $F^{-1}$ from the function $v$, we define
\begin{align}
v_F^{<}(t) = \int_{|f| \leq F} e^{itf} \hat{v}(f,\omega) df
\end{align} 

\bigskip

Let us define mean kinetic energy of random function 
\begin{align}
E = \frac{1}{2} \langle v^2 \rangle
\end{align}

Energy of harmonics with frequencies lower than $F$ are then
\begin{align}
\mathcal{E}(F) = \frac{1}{2} \langle (v_F^{<}(t))^2 \rangle
\end{align}

and energy spectrum E(f) is than defined as
\begin{align}
E(f) = \frac{d\mathcal{E}(f)}{df}.
\end{align}

Since $\mathcal{E}(f)$ is non-decreasing, we have $E(f) \geq 0$.

\bigskip

Now we can express mean kinetic energy as
\begin{align}
E = \int_0^{\infty} E(f)
\end{align}

It could be also shown that energy spectrum $E(f)$ is the Fourier transform of correlation function
\begin{align} 
E(f)  = \frac{1}{2\pi} \int_{R} e^{ifs} \Gamma(s) \,ds,
\end{align}
%what is known as \textit{Wiener-Khinchin formula} (that Einstein already used in 1914).

Direct consequence of this formula is an expression for the \textit{second order structure function}
\begin{align}\label{structf}
\langle (v(t') - v(t))^2 \rangle = 2 \, \int_{R} (1 - e^{if(t' - t)}) \, E(f) \, df.
\end{align}

\bigskip

As Kolmogorov proved in 1940, a power-law holds for energy harmonics in turbulent case:
\begin{align}
E(f) = C |f|^{-n}, ~~ C>0
\end{align}

If we substitute this expression into the structure function \ref{structf}, after tedious and hideous calculation we get

\begin{align} \label{Brown}
\begin{split}
\langle (v(t') - v(t))^2 \rangle &= C\,A_n |t' - t|^{n-1}, \\
A_n &= \int_R (1-e^{ix}) |x|^{-n} dx
\end{split}
\end{align}
for $1 < n < 3$, else the structure function diverges. For n = 2, we recover Brownian motion (formulation from the Frisch).



