\chapter{Probabilistic tools in turbulence}
In this chapter, we will remind basic concepts of probability and statistics \cite{turb} that we will use for the description of turbulent flow in the practical part.
%are frequently used in probabilistic description of turbulence. 


%So far, the most successful approach in turbulence research is by the means of statistical analysis.
%
%And life-long research of Frisch et. al., who conceived LGCA.
%
%As we saw the previous chapters, it wanted statistical physics to explain how LGCA works.
%
%LGCA and Turbulence are such a nice couple.
%
%Our thesis thrives to be a little .
%So far, the most successful approach in turbulence research is statistical analysis.
%
%\bigskip
%
%In this chapter, we introduce some basic and more advanced concepts of probability that will be handy in our research. 
%
%Following text requires prior knowledge of some probability and statistics. Basic concepts will be slightly reminded and will explain more interesting and advanced stuff in detail.
\section{Random variable}
A typical example of continuous random variable is an x-component of velocity in the turbulent flow.

Formally, random variable is defined as a map
\begin{align}
v: \Omega \rightarrow \mathbb{R}
\end{align}
from the probability space to the real numbers.
%
%A good example of random variable is the x-component of a velocity at the specific point, in the specific time. This is example of continuous random variable. In the whole section, we will consider only continuous variables and functions, as all the physical quantities of our interest are continuous.
Let us define the \textbf{cumulative probability function}
\begin{align}
F(x) = \mathrm{Prob}\left\{v(\omega) < x\right\}
\end{align}
which describes the probability that random variable $v$ takes a value smaller than $x$. 

Then, we can define \textbf{probability density function} as the derivative of cumulative probability function
\begin{align}
p(x) = \frac{\dd F(x)}{\dd x} \geq 0.
\end{align}
The inequality follows from the non-decreasing property of $F(x)$. 

Loosely speaking, $p(x)dx$ is the probability that random variable takes value between $x$ and $x+dx$.

Because the probability is normalized to one, we have
\begin{align}
\int_{\mathbb{R}} p(x) \, dx = 1.
\end{align}

Using $probability~density~function$, we can express mean value of $v$
\begin{align}
\langle v \rangle = \int_{\mathbb{R}} x \, p(x) \, dx ,
\end{align}
variance of $v$
\begin{align}
\sigma^2 = \langle v^2 \rangle = \int_{\mathbb{R}} x^2 \, p(x) \, dx ,
\end{align}
or any other statistical moment of $v$
\begin{align}
\langle v^m \rangle = \int_{\mathbb{R}} x^m \, p(x) \, dx.
\end{align}

In the following text, we suppose that the statistical moments are centered
\begin{align*}
\langle v \rangle = 0.
\end{align*}
%so the statistical moments are centered
%\begin{align*}
%\langle v^m_{central} \rangle = \langle v^m - \langle v \rangle \rangle = \langle v^m \rangle
%\end{align*}
For non-centered variables $v'$, all statements will be valid for $v = v' - \langle v' \rangle$.

\subsection{Characteristic function}

Of the special interest is the \textbf{characteristic function} of variable $v$. It is obtained by the Fourier transform of $probability~density~function$
\begin{align*}
K(z) = \int_{\mathbb{R}} e^{izv} p(x) dx = \langle e^{izv} \rangle.
\end{align*}

Let us consider random variables $v$ and $u$ with respective probability densities $f_v$ and $f_u$ and characteristic functions $K_v$ and $K_u$. The $f_{v+u}$ of the sum is given by convolution of $f_v$ and $f_u$. On the other hand, the characteristic function of the sum $K_{u+v}$ is give by product of $f_u$ and $f_v$.

\bigskip

Using characteristic function for Gaussian random variables, we will obtain several useful formulas.

\subsection{Gaussian random variables}
Gaussian random variable can be simply characterized by the \textit{characteristic function}, which simplifies to
\begin{align*}
K(z) = \langle e^{izv} \rangle = e^{-\frac{1}{2} \sigma^2 \, v^2}
\end{align*}
for the Gaussian variable.

\section{Vector random variable}
An example of the vector random variable is the velocity vector $\bm{v} = (v_x,v_y,v_z)$ of the turbulent flow.

Most of the definitions from previous section are easily generalized to vector variables, but several definitions need to be restated.

The statistical moments of the vector random variables are tensors
\begin{align}
\langle v_{i_1}, v_{i_2}, ..., v_{i_n} \rangle,
\end{align}
the most important among them being the covariance
\begin{align}
\Gamma_{ij} = \langle v_i v_j \rangle.
\end{align}
\bigskip
%Another definition worth of commentar is the definition of Gaussian random variables.
Vector random variable $\bm{v}$ is Gaussian, if scalar product $\bm{v}.\bm{c}$ if its characteristic function takes the simple form
\begin{align}
K(\bm{z}) = \langle e^{i\bm{z}.\bm{v}} \rangle = e^{-\frac{1}{2} z_i z_j \Gamma_{ij}}.
\end{align}

As we see, $K(\bm{z})$ depends only on the covariance tensor $\Gamma_{ij}$.
Starting from that, it can be shown that statistical moment of any order can be expressed as the function of covariance. For example
\begin{align}
\langle v_i v_j v_k v_l \rangle = \Gamma_{ij}\Gamma_{kl} + \Gamma_{il}\Gamma_{jk} + \Gamma_{ik}\Gamma_{jl}.
\end{align}

\section{Random function}
Concluding our velocity example, random function is the whole velocity field of the turbulent flow
$\bm{v}(t,\bm{r})$.

Formally, random function is the family of (scalar of vector-valued) random variables -- sequences indexed by continuous time and spatial variables.
% (vector random variable is sequence of random variables indexed by enumerable and finite set, e.g. $I=\big\{1,2,..N\big\}$.

\bigskip

Previous definitions are easy to upgrade to random functions, if we redefine the scalar product of random functions $u$ and $v$ as
\begin{align}
\int_{\Omega} u \, v \, dx.
\end{align}

\section{Stationary random variable}
Loosely said, $v(t)$ is stationary if $v(t)$ and $v(t+h)$ have same statistical properties.
For example, for stationary random variable $v$, we have
\begin{align*}
\langle v(t_1+h) v(t_2 + h) \rangle = \langle v(t_1) v(t_2) \rangle = \Gamma(t_1 - t_2),
\end{align*}
It implies that variance $\langle v(t)^2 \rangle = \Gamma(0)$ is time-independent.

Interestingly, difference of stationary functions is not necessarily stationary. 
For example, if $v(t)$ is stationary, then $v(2t)$ is also stationary, because its variance
\begin{align*}
\langle v(2t)^2 \rangle = \Gamma(2t - 2t) = \Gamma(0)
\end{align*}
does not depend on time.

However, their difference
\begin{align}
\langle (v(2t) - v(t))^2 \rangle = \langle v^2(t) \rangle + \langle v^2(2t) \rangle + 2\langle v(t)\,v(2t) \rangle = 2\Gamma(0) + \Gamma(2t - t)
\end{align} 
does depend on time.

\section{Spectrum of stationary random functions}

To analyze stationary random function, we can expand it in the Fourier harmonics
\begin{align}
v(t,\omega) = \int_{R} e^{itf} \hat{v}(f,\omega) df,
\end{align}
where $f$ denotes frequency.

We introduce the low-pass filter function
\begin{align}
v_F^{<}(t) = \int_{|f| \leq F} e^{itf} \hat{v}(f,\omega) df,
\end{align} 
where we filtered-out temporal scales greater than $F^{-1}$ from the function $v$.

Using the $v_F^{<}(t)$, we define the \textit{cumulative energy spectrum} as
\begin{align*}
\mathcal{E}(F) = \frac{1}{2} \langle (v_F^{<}(t))^2 \rangle.
\end{align*}
Its derivation is the \textit{energy spectrum}
\begin{align*}
E(f) = \frac{d\mathcal{E}(f)}{df}.
\end{align*}
Since $\mathcal{E}(f)$ is non-decreasing, we have
\begin{align*}
E(f) \geq 0.
\end{align*}
Integrating  $E(f)$ over all frequencies gives the total mean kinetic energy
\begin{align}
\int_0^{\infty} E(f) = \lim_{F \to \infty} \frac{1}{2} \langle (v_F^{<})^2 \rangle = \frac{1}{2} \langle v^2 \rangle.
\end{align}

It can be shown that energy spectrum $E(f)$ is the Fourier transform of correlation function
\begin{align*} 
E(f)  = \frac{1}{2 \pi} \int_{R} e^{ifs} \Gamma(s) \,ds,
\end{align*}
which is known as Wiener-Khinchin formula.

It directly implies an expression for the \textit{second order structure function}
\begin{align}\label{structf}
\langle (v(t') - v(t))^2 \rangle = 2 \, \int_{R} (1 - e^{if(t' - t)}) \, E(f) \, df.
\end{align}

As Kolmogorov proved in 1940, a power-law holds for energy harmonics in turbulent case:
\begin{align}
E(f) = C |f|^{-n}, ~~ C>0
\end{align}

Plugging the last expression in the structure function \ref{structf} leads to
\begin{align} \label{Brown}
\begin{split}
\langle (v(t') - v(t))^2 \rangle &= C\,A_n |t' - t|^{n-1}, \\
A_n &= \int_R (1-e^{ix}) |x|^{-n} dx
\end{split}
\end{align}
for $1 < n < 3$, else the structure function diverges. For n = 2, we recover Brownian motion \cite{turb}.