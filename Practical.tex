

\chapter{Practical part}

%Short review of what follows:
In the theoretical part of the thesis, we have introduced the notion of cellular automaton and basic models of lattice-gas cellular automata by which it is possible to mimic the flow of physical fluids governed by the Euler equations and, to certain extent, the Navier-Stokes equations.
We have described the algorithms behind the microdynamics of these models and emphasized the importance of the symmetries of the lattice for reproducing correct macroscopic behavior.

In this part, we implement the aforementioned algorithms in the language C++ and visualize the results using the \textit{GNU}plot software. The original results of this thesis are summarized in the points to follow.



\begin{enumerate}
\item We implemented 1D and 2D cellular automata to show their interesting properties and motivate theoretical part. These results were already included in the theoretical part and will not be discussed further.
\item We implement FCHC and PI -- the two distinct 3D LGCA introduced in theoretical part.
\item We inspect 3D flow around obstacles of various shapes.
\item We study fully-developed turbulent flow and compare its statistical pro\-perties to the Kolmogorov--Obuchov K41 theory. \cite{wolf}.

Unfortunately, this most original and scientifically most interesting part is not concluded, but it provides the basis for future research. 

\end{enumerate}
