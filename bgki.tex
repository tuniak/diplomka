\chapter{From Liouville to Navier-Stokes}
In the previous chapters, we derived hydrodynamic equations for LGCA to find out that they do not match the Eulear and Navier-Stokes equation precisely. This flaw is caused by discrete microdynamics of LGCA, which leads to the Fermi-Dirac distribution.
To treat this flaw, the Lattice-Boltzmann models have emerged.

We may say that Lattice-Boltzmann method stands on the two pillars -- Lattice gas cellular automata are the first pillar, and in this chapter we will establish the second one -- kinetic theory of gasses \cite{tong}.	
%
%Moreover, good understanding of the kinetic theory is a first step to understand the theory of lattice-gas cellular automata.

%
%Lattice-gas cellular automata are the first pillar supporting the Lattice-Boltzmann model.
%In this chapter, we will establish the second pillar -- kinetic theory of gasses \cite{tong}.


%
%On these two pillars -- the Lattice-gas cellular automata and kinetic theory, we will explain the Lattice-Boltzmann model in the subsequent chapter.
%
%As we will show in this chapter, microdynamics of the physical fluid leads to Boltzmann distribution, and starting there, we will derive correct hydrodynamic equations.
%
%The will help us to explain Lattice-Boltzmann model clearly and briefly in the subsequent chapter.
%
%
%
%
%Indeed, were pushed one ladder of abstraction higher to meet the Boltzmann distribution of real fluids.
%
%Model - which have learnt both from the lattice-gas cellular automata and 
% u
%
%In previous chapter we saw that microdynamics of LGCA lead to equilibrium occupation numbers give by Fermi-Dirac distribution.
%
%Although these models do not have discrete microdynamics anymore, they are not CA in its true sense, and they are beyond the scope of this thesis, they should be considered
%to conclude the story of LGCA.

	
\section{Liouville's equation}
Let us start by invoking the Hamiltonian for the $N$ particle system
\begin{align*}
H = \frac{1}{2m} \sum_{i=1}^N p_i^2 + \sum_{i=1}^N V(\bm{r_i}) + \sum_{i<j} U(\bm{r_i} - \bm{r_j})
\end{align*}
where $V$ is the potential of the force $\bm{F}= -\nabla V$, that affects all the particles equally, and $U$ is the potential of the $2$-particle interactions. We require that $U$ is short-ranged, meaning $U(r) \approx 0$ for $r$ of order much larger then is the atomic distance.

As usual in the framework of continuum mechanics, $N$ is a ridiculously large number, something like $N \approx 10^{23}$.

Hence, we represent the state of the system by the probability density function $f(\bm{r},\bm{p},t)$. It specifies the probability that the system is found in the vicinity of point $(\bm{r},\bm{p}) = (\bm{r_1},...,\bm{r_N},\bm{p_1},...,\bm{p_N})$.

As usual, the probability is normalized:
\begin{align*}
\int f(\bm{r},\bm{p},t) \, \dd\bm{r} \, \dd\bm{p} = 1, \quad \dd\bm{p} \dd\bm{r} = \prod_{i=1}^N \dd \bm{r_i} \dd \bm{p_i}.
\end{align*}

Since the microdynamics under consideration is deterministic, the probability density function is locally conserved and we may write its continuum equation
\begin{align} \label{protolui}
\frac{\pd f}{\pd t} + \frac{\pd}{\pd \bm{r}} (\bm{\dot{r}} f) + \frac{\pd}{\pd \bm{p}} (\bm{\dot{p}} f) = 0
\end{align}
If we substitute the Hamilton's equations 
\begin{align*}
\frac{\pd \bm{p}_i}{\pd t} = - \frac{\pd H}{\pd \bm{r}_i},~ \frac{\pd \bm{r}_i}{\pd t} = \frac{\pd H}{\pd \bm{p}_i}
\end{align*}
into the equation \ref{protolui}, we obtain the Liouville's equation
\begin{align*}
\frac{\pd f}{\pd t} + \sum_{i=1}^N \frac{\pd f}{\pd \bm{r_i}} \frac{\pd H}{\pd \bm{p_i}} - \frac{\pd f}{\pd \bm{p_i}}\frac{\pd H}{\pd \bm{r_i}} = 0.
\end{align*}
It is often written using the Poisson brackets
\begin{align} \label{lui}
\frac{\pd f}{\pd t} = \big\{ H, \, f \big\}.
\end{align}

\section{The BBGKY hierarchy}
So far, the transition to probability description did not simplified the problem much -- we are still dealing with the function of $N \approx 10^{23}$ variables.

We need to limit our ambitions and we will focus on the one-particle distribution, defined by
\begin{align} \label{1pd}
f_1(\bm{r_1},\bm{p_1},t) = N \int \prod_{i=2}^N \dd \, \bm{r_i} \dd \, \bm{p_i}
 f(\bm{r},\bm{p},t).
\end{align}
Leaving the particle with index $1$ from the product was arbitrary, because all the particles are identical. This is also the reason why $f_1$ is normalized to $N$:
\begin{align*}
\int f_1(\bm{r_1},\bm{p_1},t) \dd \bm{r_1} \dd \bm{p_1} = N.
\end{align*}
To simplify the notation, we will use $(\bm{p},\bm{r})$ instead $(\bm{p_1},\bm{r_1})$ from now on.

For many purposes, the function $f_1$ captures all we need to know.
For example, it gives us the average density of particles at the spatial point $\bm{r}$
\begin{align} \label{partdens}
n(\bm{r},t) = \int f_1(\bm{r},\bm{p},t) \dd \bm{p},
\end{align}
the average velocity of the particle
\begin{align*}
u(\bm{r},t) = \int \frac{\bm{p}}{m} f_1(\bm{r},\bm{p},t) \dd \bm{p},
\end{align*}
and the energy flux
\begin{align*}
\varepsilon(\bm{r},t) = \int \frac{\bm{p}}{m} E(\bm{p}) f_1(\bm{r},\bm{p},t) \dd \bm{p}.
\end{align*}
To derive the equation governing the evolution of $f_1$, let us see how it varies in time:
\begin{align*}
\frac{\pd f_1}{\pd t} = N \int \prod_{i=2}^N \dd \, \bm{r_i} \dd \, \bm{p_i} \frac{\pd f}{\pd t} = N \int \prod_{i=2}^N \dd \, \bm{r_i} \dd \, \bm{p_i} \big\{ H, \, f \big\}.
\end{align*}
Arranging the terms on the right, we can turn this equation in the form
\begin{align} \label{boltzman}
\frac{\pd f_1}{\pd t} = \big\{ H_1, f_1 \big\} + \Big(\frac{\pd f_1}{\pd t} \Big)_{coll},
\end{align}
where
\begin{align*}
H_1 = \frac{p^2}{2m} + V(\bm{r}).
\end{align*}
We see that the evolution of $f_1$ is governed by the Liouville's equation plus the extra term, the \textit{collision integral}
\begin{align*}
\Big(\frac{\pd f_1}{\pd t} \Big)_{coll} = N(N-1) \int \dd \bm{r_2} \dd \bm{p_2} 
\frac{\pd U(\bm{r} - \bm{r_2})}{\pd \bm{r}} . \frac{\pd}{\pd \bm{p}} \int \prod_{i=3}^N \dd \bm{r_i} \dd\bm{p_i} f(\bm{r},\bm{r_2},...,\bm{p},\bm{p_2},...,t).
\end{align*}
Not surprisingly, it is not determined by the 1-particle distribution, because it does not contain the relation of one particle to all other particles. Some of that information is carried by the \textit{2-particle distribution}
\begin{align*}
f_2(\bm{r},\bm{r_2},\bm{p},\bm{p_2},t) := N(N-1)\int f(\bm{r},\bm{r_2},\bm{r_3},...,\bm{p},\bm{p_2},\bm{p_3},...,t) \prod_{i=3}^N \dd \bm{r_i} \dd \bm{p_i}.
\end{align*}
Using the 2-particle distribution, we can rewritten the collision integral as
\begin{align*}
\Big(\frac{\pd f_1}{\pd t} \Big)_{coll} = \int \dd \bm{r_2} \dd \bm{p_2} \frac{\pd U(\bm{r} - \bm{r_2})}{\pd \bm{r}} . \frac{\pd f_2}{\pd\bm{p}}.
\end{align*}
We see that to follow the evolution of one-particle distribution $f_1$, we also need to know something about the two-particle distribution $f_2$. And again, to know the evolution of the two-particle distribution, we repeat the same procedure as before and find the Liouville-like equation for $f_2$ with additional term containing the three-particle distribution $f_3$.

In general, the evolution of \textit{n-particle distribution} is described by
\begin{align} \label{bgky}
\frac{\pd f_n}{\pd t} = \big\{ H_n, f_n \big\} + \sum_{i=1}^n \int \dd \bm{p_{n+1}} \dd \bm{r_{n+1}} \frac{\pd U(\bm{r_i} - \bm{r_{n+1}}}{\pd \bm{r_i}}).\frac{\pd f_{n+1}}{\pd \bm{p_i}},
\end{align}
where the Hamiltonian 
\begin{align*}
H_n = \sum_{i=1}^n \Big(\frac{\bm{p_i}}{2m} + V(\bm{p_i})\Big) + \sum_{i<j\leq n} U(\bm{r_i} - \bm{r_j})
\end{align*}
includes external forces $V$ and interactions between the $n$ particles.

The set of equations \ref{bgky} is called the \textit{BBGKY hierarchy}.
Instead of the Liouville's equation governing the function of $N \approx 10^{23}$ variables, we have now the set of $10^{23}$ functions.

However, it is more advantageous to work with the hierarchy of equations \ref{bgky}, as it is ready to be approximated. Depending on the particular problem at hands, we decide which terms are important and which terms can be ignored, thus truncating the problem to something manageable.

\section{Boltzmann equation}
Let us derive the most simple and useful of these truncations, the Boltzmann equation
\begin{align*}
\frac{\pd f_1}{\pd t} = \big\{ H_1, f_1 \big\} + \Big(\frac{\pd f_1}{\pd t} \Big)_{coll},
\end{align*}
in the closed form, where collision integral is the expression of $f_1$ alone.
%with the collision integral that is the closed expression for the $f_1$ alone.
%OPRAVIT
%that we already saw. The collision integral depends on the $f_2$. We would like to get rid of this dependence.
%, the closed equation for the $f_1$ alone.
%
%We already saw its general form
%but the collision integral was dependent on $f_2$.

To proceed, let us define two time scales:
the time between particle collisions $\tau$, and the time of the collision itself, $\tau_{coll}$.
In a dilute gas, we have
\begin{align*}
\tau >> \tau_{coll},
\end{align*}
so most of the time, $f_1$ follows the Hamiltonian evolution, with occasional, but abrupt perturbations by the collision. 

What is the rate at which the collision happens?

Let the two particles with momenta $\bm{p_1}$ and $\bm{p_2}$ collide at the point $\bm{r}$, and during the collision, they gain velocities $\bm{p'_1}$ and $\bm{p'_2}$.
We can define the rate of the collision by
\begin{align*}
Rate = \omega(\bm{p_1},\bm{p_2} | \bm{p'_1}, \bm{p'_2}) f_2(\bm{r},\bm{r},\bm{p_1},\bm{p_2}),
\end{align*}
where $\omega$ carries information about the dynamics of the collision, and can be computed from the inter-atomic potential $U$ \cite{tong}.

Next, we focus on the collisions where
\begin{enumerate}
\item one of the particles has specific momentum $\bm{p_1}$ before collision, and gains some unspecified momentum $\bm{p'_1}$ in the collision, or
\item the particle had unspecified momentum $\bm{p'_1}$ before the collision, but gains the specific momentum $\bm{p_1}$ in the collision.
\end{enumerate}
The collision integral governing these collisions contains two terms:
\begin{align*}
\Big( \frac{\pd f_1}{\pd t} \Big)_{coll}= \int \big[ \omega(\bm{p'_1},\bm{p'_2} | \bm{p_1}, \bm{p_2}) f_2(\bm{p'_1},\bm{p'_2}) - \omega(\bm{p_1},\bm{p_2} | \bm{p'_1}, \bm{p'_2}) f_2(\bm{p_1}, \bm{p_2}) \big] \dd \bm{p_2} \dd \bm{p'_1} \dd \bm{p'_2}.
\end{align*}
Considering the symmetries of the process (invariance under time reversal and parity), we have
\begin{align} \label{symm} 
\omega(\bm{p'_1},\bm{p'_2} | \bm{p_1}, \bm{p_2}) = \omega(\bm{p_1},\bm{p_2} | \bm{p'_1}, \bm{p'_2}).
\end{align}
Finally, in order to express the collision integral in terms of $f_1$ instead of $f_2$, we make a non-trivial assumption, that velocities of colliding particles are uncorrelated:
\begin{align} \label{chaos}
f_2(\bm{r},\bm{r},\bm{p_1},\bm{p_2}) = f_1(\bm{r},\bm{p_1})\,f_1(\bm{r},\bm{p_2}).
\end{align}
This assumption is known as a \textit{molecular chaos hypothesis}, and seems quite innocent. However, by supposing that particles are uncorrelated before the collision, but they inevitably become correlated in the collision, we introduce the arrow of time on the scene.

Finally, using \ref{symm} and \ref{chaos}, we can write the collision integral in the form
\begin{align} \label{colint}
\Big( \frac{\pd f_1}{\pd t} \Big)_{coll}= \int \omega(\bm{p'_1},\bm{p'_2} | \bm{p_1}, \bm{p_2}) \big[f_1(\bm{r},\bm{p'_1})f_1(\bm{r},\bm{p'_2}) - f_1(\bm{r},\bm{p_1})f_1(\bm{r},\bm{p_2}) \big] \dd \bm{p_2} \dd \bm{p'_1} \dd \bm{p'_2}
\end{align}
and we can express the \textit{Boltzmann equation}
\begin{align} \label{Boltz}
\frac{\pd f_1}{\pd t}  = \big\{ H_1, f_1 \big\} + \Big(\frac{\pd f_1}{\pd t} \Big)_{coll}
\end{align}
in the closed form.

\section{Equilibrium and detailed balance}
The \textit{equilibrium distribution} is defined such that it does not explicitly depend on the time:
\begin{align*}
\frac{\pd f_1^{eq}}{\pd t} = 0.
\end{align*}
Using the Boltzmann equation, we can express this condition equivalently as
\begin{align*}
\big\{ f_1, H_1 \big\} + \Big( \frac{\pd f_1}{\pd t} \Big)_{coll} = 0.
\end{align*}
If we restrict ourselves to the case with vanishing external force
%This is automatically satisfied if $f$ is the function of $H$, the simple example being the Boltzmann distribution
%\begin{align*}
%f \approx e^{-\beta H}.
%\end{align*}
\begin{align*}
V(\bm{r}) = 0,
\end{align*}
any function of momentum commutes with Hamiltonian and the streaming term $\big\{ f_1, H_1 \big\}$ vanishes.
According to \ref{colint}, to make the collision integral vanish, we need to satisfy the \textit{detailed balance} condition
\begin{align} \label{detailedbalance}
f_1^{eq}(\bm{r},\bm{p_1})f_1^{eq}(\bm{r},\bm{p_2}) = f_1^{eq}(\bm{r},\bm{p'_1})f_1^{eq}(\bm{r},\bm{p'_2}),
\end{align}
or equivalently
\begin{align*}
\log(f_1^{eq}(\bm{r},\bm{p_1})) + \log(f_1^{eq}(\bm{r},\bm{p_2})) = \log(f_1^{eq}(\bm{r},\bm{p'_1})) + \log(f_1^{eq}(\bm{r},\bm{p'_2})).
\end{align*}

According to this equation, the sum of logarithms is conserved in the collision.

As we know, such quantities are either momenta or energy, hence the logarithms are the linear combinations of energy and momenta
\begin{align*}
log(f_1^{eq}(\bm{r},\bm{p})) = \beta(\mu - E(\bm{p}) + \bm{u} \cdot \bm{p}),
\end{align*}
where
\begin{align*}
E(\bm{p}) = \frac{1}{2m} p^2
\end{align*}
and $\beta,\mu$ and $\bm{u}$ are constants.
We need to set the constant $\mu$ such that $f_1$ is normalized, which gives us
\begin{align} \label{mbt}
f_1^{eq} = \frac{N}{V}\big(\frac{\beta}{2\pi m}\big)^{3/2} e^{-\beta m (\bm{v} - \bm{u})^2/2}.
\end{align}
Supposing that $\beta$ is the inverse temperature, this is the Maxwell-Boltzmann distribution.

If we forget about the streaming term $\big\{ H_1, f_1 \big\}$, then the detailed balance condition \ref{detailedbalance} leads to the much larger class of solutions, so called \textit{local equilibrium distributions}
\begin{align} \label{localequil}
f_1^{loc}(\bm{r},\bm{p},t) = n(\bm{r},t) \Big(\frac{\beta(\bm{r},t)}{2\pi m}\Big)^{3/2} \exp \big(-\beta(\bm{r},t)\frac{m}{2}[\bm{v} - \bm{u}(\bm{r},t)]^2 \big)
\end{align}
where $n,\beta$ and $\bm{u}$ are promoted to the functions.
By inserting these distributions into the Boltzmann equation, the collision integral vanishes, but the streaming term stays.

\section{Hydrodynamics}
On the most coarse-grained level, the system in local equilibrium is described by hydrodynamics, with the parameters that do not relax to their equilibrium values immediately, but vary slowly in space and time.
%Just like the termodynamics describes systems in the equilibrium, hydrodynamics describes systems in the local equilibrium, with the parameters that vary slowly in space and time.
%These are quantities that do not relax to their equilibrium values immediately, but change on the much slower scale. 
In this section we will show that these parameters are density $\rho(\bm{r},t)$, temperature $T(\bm{r},t)$ and velocity $\bm{u}(\bm{r},t)$ (thus showing that the functions in \ref{localequil} have their usual meaning).

Let us have an arbitrary function $A(\bm{r},\bm{u})$.
To see how does the function $A$ vary in space, we integrate over the momentum space
\begin{align} \label{averg}
\langle A(\bm{r},t) \rangle = \frac{\int A(\bm{r},\bm{p}) f_1(\bm{r},\bm{p},t) \dd \bm{p}}{\int f_1(\bm{r},\bm{p},t) \dd \bm{p}}.
\end{align}
We are already familiar with the denominator -- it is the particle density function, so we can write
\begin{align*}
\langle A(\bm{r},t) \rangle = \frac{1}{n(\bm{r},t)} \int A(\bm{r},\bm{p}) f_1(\bm{r},\bm{p},t) \dd \bm{p}.
\end{align*} 

Note that the resulting function is a function of space and time, since we computed average only over the momentum variables.

%To see how this averaged function evolve in time
We are interested in the functions $A$ that vanish, when integrated against the collision integral
\begin{align*}
\int A(\bm{r},\bm{p})  \Big( \frac{\pd f_1}{\pd t} \Big)_{coll} \dd \bm{p} = 0.
\end{align*}
Intuitively, this condition is required because we are interested in the quantities that vary slowly, but the collisions are abrupt processes, and the term involving collision integral vary fast.

Substituting the expression for the collision integral \ref{colint} to the last equation, we get
\begin{align*}
\int A(\bm{r},\bm{p_1}) \omega(\bm{p'_1},\bm{p'_2} | \bm{p_1}, \bm{p_2}) \big[f_1(\bm{r},\bm{p'_1})f_1(\bm{r},\bm{p'_2}) - f_1(\bm{r},\bm{p_1})f_1(\bm{r},\bm{p_2}) \big] \dd \bm{p_1} \dd \bm{p_2} \dd \bm{p'_1} \dd \bm{p'_2} = 0.
\end{align*}
Considering the symmetries of $\omega$, this condition is equivalent to
\begin{align*}
\begin{split}
\int \omega(\bm{p'_1},\bm{p'_2} | \bm{p_1}, \bm{p_2}) \big[f_1(\bm{r},\bm{p'_1})f_1(\bm{r},\bm{p'_2}) - f_1(\bm{r},\bm{p_1})f_1(\bm{r},\bm{p_2}) \big] \\ \cross \big[A(\bm{r},\bm{p_1}) + A(\bm{r},\bm{p_2}) - A(\bm{r},\bm{p'_1}) - A(\bm{r},\bm{p'_2}) \big] \dd \bm{p_1} \dd \bm{p_2} \dd \bm{p'_1} \dd \bm{p'_2} = 0
\end{split},
\end{align*}
which holds if
\begin{align*}
A(\bm{r},\bm{p_1}) + A(\bm{r},\bm{p_2}) = A(\bm{r},\bm{p'_1}) + A(\bm{r},\bm{p'_2}).
\end{align*}

So $A$ is a quantity that is conserved during the collision, which is true for the energy, momenta, and trivially, for a constant function, e.g. $A=1$.
These are the \textit{collision invariants} of our interest.

If we plug any collision invariant to the Boltzmann equation \ref{Boltz} and integrate over the momenta variables, the term involving collision integral vanishes
\begin{align*}
\int A(\bm{r},\bm{p}) \Big( \frac{\pd }{\pd t} + \frac{\bm{p}}{m}\cdot \frac{\pd}{\pd \bm{r}} + \bm{F} \cdot \frac{\pd}{\pd \bm{p}} \Big) f_1(\bm{r},\bm{p},t) \dd \bm{p} = 0
\end{align*} 
Because $A$ do not explicitly depend on time, we can put it inside the time derivative, and using integration by part on other terms \textit{per partes} leads to
\begin{align*}
\frac{\pd}{\pd t} \int A f \dd\bm{p} + \frac{\pd}{\pd \bm{r}} \cdot \int \frac{\bm{p}}{m} A f \dd \bm{p} - \int \frac{\bm{p}}{m} \cdot \frac{\pd A}{\pd \bm{r}} f \dd \bm{p} - \int \bm{F} \cdot \frac{\pd A}{\pd \bm{p}}f \dd \bm{p} = 0
\end{align*}
Finally, applying the expression for the average \ref{averg} on this equation, we can write
\begin{align} \label{master}
\frac{\pd}{\pd t} \langle n A \rangle + \frac{\pd}{\pd \bm{r}} \cdot \langle n \bm{u} A \rangle - n \langle\bm{u} \cdot \frac{\pd A}{\pd \bm{r}} \rangle - n \langle \bm{F} \cdot \frac{\pd A}{\pd \bm{u}} \rangle = 0,
\end{align}
which is our master equation governing the evolution of the collisional invariant.
In between, we have introduced notation for the velocity
\begin{align*}
\bm{v} = \frac{\bm{p}}{m},
\end{align*}
and now we introduce the average velocity
\begin{align*}
\bm{u} = \langle \bm{v} \rangle.
\end{align*}


Substituting the constant $A=1$ into the master equation yields
\begin{align*}
\frac{\pd n}{\pd t} + \frac{\pd}{\pd \bm{r}} \cdot (n \bm{u}). 
\end{align*}
Multiplying by particle mass $m$, this is the continuity equation.

Substituting the next collisional invariant, the momentum $A=m\bm{v}$, yields
\begin{align} \label{conmom}
\frac{\pd}{\pd t} (mn u_i)  + \frac{\pd}{\pd r_j}\langle mn v_i v_j \rangle - \langle n F_i \rangle = 0
\end{align}
If we expand the middle term like
\begin{align*}
\langle v_i v_j \rangle = \langle (u_j - v_j) (u_i - v_i) \rangle + v_i\, v_j
\end{align*}
and if we define the quantity
\begin{align*}
P_{ij} = P_{ji} = \rho \langle (u_j - v_j) (u_i - v_i) \rangle
\end{align*}
known as the pressure tensor,
we can rewrite the equation \ref{conmom} as
\begin{align} \label{mmcl}
\rho \Big(\frac{\pd}{\pd t} + u_j \frac{\pd}{\pd x_j} \Big) u_i = \frac{\rho}{m} F_i - \frac{\pd}{\pd x_j}P_{ij}.
\end{align}
This is the familiar form of the momentum conservation law. We can interpret it according to the Newton's law -- the left-hand side is the acceleration of the fluid element, and the right-hand side involves the external force $\bm{F}$ and the internal pressure of the fluid
%The collision integral reflects the rate of the collisions.

If we evaluate the pressure tensor on the equilibrium distribution, we find out that it is proportional to the temperature
\begin{align*}
P_{ij} = n k_B T \delta_{ij},
\end{align*} 
which is the ideal gas law.

\bigskip

The last collisional invariant is the kinetic energy of particles.
Instead of absolute kinetic energy, we shall use the relative kinetic energy
\begin{align*}
A  = \frac{1}{2} m (\bm{u} - \bm{v})^2,
\end{align*}
so when we substitute it into our master equation \ref{master}, we get the compact form
\begin{align} \label{cone}
\frac{1}{2} \frac{\pd}{\pd t} \langle \rho(\bm{v} - \bm{u}) \rangle + \frac{1}{2} \frac{\pd}{\pd r_i} \langle \rho v_i (\bm{v} - \bm{u}) \rangle - \rho \langle v_i \frac{\pd u_j}{\pd r_i}(\bm{v} - \bm{u})^2 \rangle = 0.
\end{align}
Using the idea of equipartition, we can define the temperature for the non-equilibrium system
\begin{align*}
\frac{3}{2}k_B T(\bm{r},t) = \frac{1}{2} m \langle (\bm{v} - \bm{u}(\bm{r},t))^2. \rangle
\end{align*}

Let us define another quantity, the \textit{heat flux}
\begin{align*}
\frac{1}{2} m \rho \langle (v_i - u_i)(\bm{v} - \bm{u})^2 \rangle.
\end{align*}
Using the definition of heat flux and pressure tensor, we rewrite the equation \ref{cone} as
\begin{align*}
\frac{1}{2}m \rho \langle v_i (\bm{v} - \bm{u})^2 \rangle = q_i + \frac{3}{2}\rho u_ik_B T,
\end{align*}
Since the pressure tensor is symmetric, we can replace $\pd u_j / \pd r_i$ with the symmetric \textit{rate of strain}
\begin{align*}
U_{ij} = \frac{1}{2}\big( \frac{\pd u_j}{\pd r_i} + \frac{\pd u_i}{\pd r_j} \big).
\end{align*}
Further, using the continuity equation, we obtain the conservation of energy in its usual form
\begin{align*}
\rho \big( \frac{\pd}{\pd t} + u_i \frac{\pd}{\pd r_i} \big) k_B T + \frac{2}{3} \frac{\pd q_i}{\pd r_i} + \frac{2m}{3} U_{ij} P_{ij} = 0.
\end{align*}

Finally, we have the three equations, describing the time evolution of the particle density $n$, momentum $\bm{u}$ and the temperature $T$.
Although they hold for any distribution $f_1$, they do not constitute the closed set of equations -- equation for $n$ depends on $\bm{u}$, equation for momentum depends on the pressure tensor $P_{ij}$ and the equation for temperature depends on the heat flux $q$. To determine any of these, we need to compute the distribution $f_1$ from the Boltzmann equation, which is not easy.

Since we are looking for the solution that varies slowly, the first approximation would be the  local equilibrium distribution \ref{localequil}, for which collision integral vanishes.
%
%Since local equilibrium distribution is not solution of the Boltzmann equation, because streaming term is non-zero, we will see later on what we are missing.
Using this distribution, we can compute that
\begin{align*}
P_{ij} = k_b n(\bm{r},t) T(\bm{r},t) \delta_{ij} = P(\bm{r},t) \delta_{ij}
\end{align*} and
\begin{align*}
q = 0.
\end{align*}
Inserting these terms into the conservation laws, the continuity equation stays unchanged, we just rewrite it using $\rho = m\,n$:
\begin{align} \label{cunt}
\big(\frac{\pd}{\pd t} + u_j \frac{\pd}{\pd r_j} \big) \rho + \rho \frac{\pd u_i}{\pd r_i}.
\end{align}
However, the momentum conservation \ref{mmcl} turns into the Euler equation
\begin{align} \label{ell}
\big(\frac{\pd}{\pd t} + u_j \frac{\pd}{\pd r_j} \big) u_i  + \frac{1}{\rho}  \frac{\pd P}{\pd x_i} = \frac{F_i}{m}.
\end{align}
However, we are still missing dissipation in these equations. To show it, we can combine equations \ref{cunt} and \ref{ell} to obtain
\begin{align*}
\big( \frac{\pd}{\pd t} + u_j \frac{\pd}{\pd r_j} \big) (\rho T ^{-2/3}) = 0
\end{align*}
This equation implies that $\rho T^{-2/3}$ is constant along the streamlines, which means that the motion of the fluid is adiabatic -- not increasing entropy. Hence we have no mechanism for the fluid to return to the equilibrium yet.

It is easy to see what we are missing. We chose local equilibrium as an approximation for the distribution $f_1$, so that the collision integral vanishes, but it does not solve the streaming term. However, if we stick to the long wave-length variations in the velocity, we need to add only little correction term
\begin{align} \label{cord}
f_1 = f_1^{(0)} + \delta f_1.
\end{align}
The correction $\delta f_1$ contributes to the collision integral
\begin{align*}
\Big( \frac{\pd f_1}{\pd t} \Big)_{coll}= \int \omega(\bm{p'_1},\bm{p'_2} | \bm{p_1}, \bm{p_2}) \big[ f_1(\bm{p'_1})f_1(\bm{p'_2}) - f_1(\bm{p_1})f_1(\bm{p_2}) \big] \dd \bm{p_2} \dd \bm{p'_1} \dd \bm{p'_2}
\end{align*}
where we neglected the quadratic terms $O(\delta f_1^2)$ and used the fact that $f_1^{(0)}$ vanish in the collision integral.

To proceed further, the proper way would be to use \textit{Chapman-Enskog expansion} of $\delta f_1$ to treat the collision operator properly, but there is an easier way to progress, the so called $BGK$ \footnote{Bhatnagar-Gross-Krook} approximation. We simply set the collision integral to be
\begin{align} \label{bgk}
\Big( \frac{\pd f_1}{\pd t} \Big)_{coll} = - \frac{\delta f_1}{\tau},
\end{align}
where $\tau$ is the relaxation time, and we take it to be a constant (although in general, it might be function of momentum).
With this replacement, the Boltzmann equation changes to
\begin{align*}
\frac{\pd (f_1^{(0)} + \delta f_1)}{\pd t} - \big\{ H_1, f_1^{(0)} + \delta f_1 \big\} = -\frac{\delta f_1}{\tau}
\end{align*}

\section{Where is the viscosity?}
Let us consider the flow of the fluid with the velocity gradient $\pd u_x / \pd z \neq 0$.
According to the Newton's law, the shear viscosity is associated with the flux of $x$-momentum to the $z$-direction:
\begin{align} \label{newt}
\Pi_{xz} = -\eta \frac{\pd u_x}{\pd r_z},
\end{align}
which is the off-diagonal component of the pressure tensor
\begin{align*}
P_{xz} = \rho \langle (v_x - u_x) (v_z - u_z) \rangle.
\end{align*}

As we already know, the local equilibrium distribution gives us the diagonal pressure tensor, but if we use the corrected distribution, we have
\begin{align}
P_{ij} = P \delta_{ij} + \Pi_{ij},
\end{align}
with the additional term $\Pi_{ij}$, which is a non-diagonal \textit{stress tensor}
\begin{align*}
\Pi_{ij} = \frac{m \tau \rho}{k_B T} U_{kl} = \Big[ \langle v_i v_j v_k v_l \rangle_0 - \frac{1}{3} \delta_{kl} \langle v_i v_j v^2 \rangle_0 \Big].
\end{align*}
In fact, the $\Pi_{ij}$ is traceless, because
\begin{align*}
\langle v^2 v_k v_l \rangle_0 = \delta_{kl} \langle v_i v_j v^2 \rangle_0. 
\end{align*}
and it depends linearly on the $U_{kl}$.
These two properties determine the form of the tensor
\begin{align} \label{stresst}
\Pi_{ij} = -2 \mu \big( U_{ij} - \frac{1}{3} \delta_{ij} \nabla . \bm{u} \big).
\end{align}
For $\pd u_x / \pd z \neq 0$, we can evaluate
\begin{align*}
\Pi_{xz} = \frac{2m\tau \rho}{15k_B T} U_{xx} \langle v^4 \rangle_0.
\end{align*}
Comparing it to the Newton's law \ref{newt}, we get an expression for the shear viscosity
\begin{align} \label{shearv}
\eta = n k_B T \tau.
\end{align}

Note that this expression is just an estimation, as it relies on the relaxation time approximation $\tau$.

\section{Navier-Stokes equations}

The corrected distribution \ref{cord} does not change the density equation
\begin{align*}
\frac{\pd \rho}{\pd t} + \nabla . (\rho \bm{u}) = 0,
\end{align*}
but due to the non-diagonal stress tensor \ref{stresst}, the momentum equation transforms to the Navier-Stokes equation
\begin{align}
\Big(\frac{\pd}{\pd t} + \bm{u} \cdot \nabla \Big) \bm{u} = \frac{\bm{F}}{m} - \frac{1}{\rho} \nabla P + \frac{\eta}{\rho} \nabla^2 \bm{u} + \frac{\eta}{3 \rho}\nabla(\nabla \cdot \bm{u})
\end{align}
%Since $\nabla \kappa \approx 0$ and $U_{ij} \Pi_{ij} \approx 0$ are small at the order we are working at, we can drop them and the equation reduces to
%\begin{align*}
%aa
%\end{align*}
Although our derivation relied on the dilute gas approximation, the Navier-Stokes equation are more general then that and can be considered to be the most general expression for the momentum transport \cite{tong}.

%This force K will be neglected in the subsequent text, because external force is not supposed to influence dynamics of collisions.
%The right hand side of Boltzmann equation, the $Q(f,f)$, is the collision integral.
%It is quite complicated expression
%and it is the major difficulty in dealing with Boltzmann equation.
%But we do not need it that precisely, as two-particle collisions do not significantly influence experimentally measured quantities.
%Therefore, it is often approximated.
%Before we show the standard way to approximate it (so called BGK approximation), we need to present the properties of $Q(f,f)$ that need to be preserved.

%Interestingly, function 
%\begin{align} \label{nulling}
%\phi = \exp (a + \bm{b}. \bm{v} + c \, v^2)
%\end{align}
%is the solution of
%\begin{align}
%Q(f,f) = 0,
%\end{align}
%if $c$ is negative. Notice, that famous Maxwell function (or Maxwell distribution, or simply Maxwellian) is the spacial kind of such function \ref{nulling} 
%\begin{equation} \label{Maxwell}
%f^M(\bm{x,v,}t) = n(\frac{m}{2 \pi k_B T})^{3/2} e^{-\frac{m}{2k_B T}(\bm{v-u})^2}.
%\end{equation}
%
%Now we are ready to state what is BGK approximation of collision integral $Q(f,f)$.
%\section{BGK approximation}
%Instead of $Q(f,f)$, that is so complicated we do not even state it, we will use simpler operator 
%\begin{align}
%J(f) = w[f^M(\bm{x,v}) - f(\bm{x,v}]
%p\end{align}
%
%We can easily see that this approximation have two properties:
%\begin{enumerate}
%\item It expresses the tendency of $f$ towards Maxwellian distribution \ref{Maxwell} (because the time derivative of $f$ has sign towards $f^M$).
%\item It conserves the collision invariants \ref{invariants}
%\begin{align}
%\int \psi_k J(f) d^3 x d^3 v = 0
%\end{align}
%\end{enumerate}
%
%
%\section{Chapman-Enskog series}
%We are already familiar with this kind of expansion, now we will show it in the general context.
%In some respect this expansion is quite peculiar and deserves few comments before we suck the life out of it.
%
%\bigskip
%
%The distribution function f, that is explicitly time-dependent, will be expanded in the series of $f^{(n)}$, that have only implicit time-dependence. However, gas relaxes very fast, in time of order $10^{-11}$ seconds, towards time-independent distributions, hence this expansion is generally valid.
%
%
%The expansion parameter $\epsilon$ depended on the peculiarities of our model in previous chapters. 
%In this section, we will take $\epsilon$ equal to Knudsen number
%\begin{align}
%K_n = \frac{\lambda}{L},
%\end{align}
%that is ratio of $\lambda$, the mean free path of particle (the mean distance between two successive collisions) and spatial characteristic scale $L$ (for example, size of the obstacle in the flow, diameter of the flow etc.).
%
%\bigskip
%
%And the most interesting point comes last - the Chapman-Enskog series diverges.
%
%We are interested only in first two orders of the expansion, but higher order-approximation that lead to Burnett equations are problematic and were never applied systematically (for further details see Cercignani 1988 and 1990).
%
%
%\section{Conservation laws}
%What happens if we multiply Boltzmann equation with the collision invariants \ref{invariants} and integrate it?
%
%Since the collision integral in the right-hand side gets killed-off (by definition of invariants), 
%the equation reads
%\begin{equation}
%\int \psi_k(\partial_t + v_{\alpha}\partial_{\alpha}) f(\bm{x},\bm{v},t) d^3v = 0.
%\end{equation}
%Substituting $\psi_0 = 1$, $(\psi_1,\psi_2,\psi_3) = \bm{v}$ and $\psi_4 = v^2$, we get the general form of conservation laws:
%
%\begin{align}
%\begin{split}
%\partial_t\rho + \partial_{\alpha}(\rho u_{\alpha}) &= 0 \\
%\rho \partial_t u_{\alpha} + \rho u_{\beta}\partial_{\beta}u_{\alpha} &= - \partial_{\alpha}\hat{P}_{\alpha\beta} \\
%\rho \partial_t \Theta + \rho u_{\beta}\partial_{\beta}\Theta &= -\frac{2}{3} \partial_{\alpha}q_{\alpha} - \frac{2}{3} \hat{P}_{\alpha\beta} \Lambda_{\alpha\beta}
%\end{split}
%\end{align}
%where
%\begin{align}
%\begin{split}
%n(\bm{x}, t) &= \int f(\bm{x},\bm{v},t) \, d^3 v  \\
%p(\bm{x},t) &= m \, n(\bm{x},t) \\
%\rho u_{\alpha}(\bm{x},t) &= m \int v_{\alpha} f(\bm{x},\bm{v},t) \, d^3 v \\
%\Theta(\bm{x},t) &= k_B \, T(\bm{x},t) = \frac{m}{3n} \int (v_{\alpha} - u_{\alpha})(v_{\alpha} - u_{\alpha}) \, f(\bm{x},\bm{v},t) \, d^3 v \\
%\Lambda_{\alpha\beta} &= \frac{m}{2} (\pd_{\beta}u_{\alpha} + \pd_{\alpha} u_{\beta}) \\
%\hat{P}_{\alpha\beta} &= m \int (v_{\alpha} - u_{\alpha})(v_{\beta} - u_{\beta}) \, f(\bm{x}, \bm{v}, t) \, d^3v \\
%q_{\alpha}(\bm{x},t) &= \frac{m^2}{2} \int (v_{\alpha} - u_{\alpha})(v_{\beta} - u_{\beta}) (v_{\beta} - u_{\beta}) \, f(\bm{x}, \bm{v}, t) \, d^3v
%\end{split}
%\end{align}
%
%\section{Euler equation}
%
%$f^M$ into conservation laws:
%$\partial_t \rho + \partial_{\alpha}(\rho u_{\alpha}) = 0$ ... continuity equation\\
%$\rho \partial_t u_{\alpha} + \rho u_{\beta}\partial_{\beta}u_{\alpha} = - \partial_{\alpha}P$ ... Euler equation,\\
%where $P = n\Theta = n k_B T$\\
%$\rho \partial_t \Theta + \rho u_{\beta}\partial_{\beta}\Theta = -\frac{1}{c_V} \Theta \partial_{\alpha} u_{\alpha}$\\
%
%Chapman-Enskog:\\
%$f = f^0 + \epsilon f^1 + \epsilon^2 f^2 + ... = $
%
%It is special in every sense of word, we will introduce some of its useful properties: