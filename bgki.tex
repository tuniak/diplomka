\chapter{Hydrodynamic equations}
Let us start by invoking hamiltonian for the N particle system
\begin{align*}
H = \frac{1}{2m} \sum_{i=1}^N p_i^2 + \sum_{i=1}^N V(r_i) + \sum_{i<j} U(r_i - r_j)
\end{align*}
where $V$ is the potential of the force $F=\nabla V$, that effects all the particles equally, and $U$ is the potential of the $2$-particle interactions. We require that $U$ is short-ranged, meaning $U(r) \approx 0$ for $r>>d$, where $d$ is the atomic distance scale.

As usual in the framework of continuum mechanics, $N$ is a ridiculously large number, something like $N \approx 10^{23}$.

Hence, we represent the state of the system by the probability density function $f(\bm{x},\bm{u},t)$. It specifies the probability that the system is found in the vicinity of point $(\bm{x},\bm{u})$.

As usual, the probability is normalized:
\begin{align*}
\int f(\bm{x},\bm{u},t) \, \dd\bm{x} \, \dd\bm{u}
\end{align*}

Since the microdynamics under consideration of deterministic, the probability density function is locally conserved and we may write its continuum equation
\begin{align} \label{protolui}
\frac{\pd f}{\pd t} + \frac{\pd}{\pd \bm{x}} (\bm{\dot{x}} f) + \frac{\pd}{\pd \bm{u}} (\bm{\dot{u}} f) = 0
\end{align}
Notice that this equation can be written using the material derivative
\begin{align*}
\frac{D}{D\,t} = \frac{\pd}{\pd t} + \nabla
\end{align*}
By inserting the Hamilton's equations 
\begin{align*}
\frac{\pd \vec{p}_i}{\pd t} = - \frac{\pd H}{\pd \vec{r}_i},~ \frac{\pd \vec{r}_i}{\pd t} = \frac{\pd H}{\pd \vec{p}_i}
\end{align*}
into the equation \ref{protolui}, we obtain the Liouville's equation
\begin{align*}
\frac{\pd f}{\pd t} + \frac{\pd f}{\pd \bm{x}} \frac{\pd H}{\pd \bm{u}} - \frac{\pd f}{\pd \bm{u}}\frac{\pd H}{\pd \bm{x}} = 0.
\end{align*}
It can be written in the shorter form using the Poisson brackets
\begin{align*}
\frac{\pd f}{\pd t} = \big\{ H, \, f \big\}.
\end{align*}

Let us make a little side-step.

The \textit{equilibrium distribution} is defined as not explicitly depending on time:
\begin{align*}
\frac{\pd f}{\pd t} = 0.
\end{align*}
That also means
\begin{align*}
\big\{ H, f \big\} = 0.
\end{align*}
This is automatically satisfied if $f$ is the function of $H$, for example the Boltzmann distribution
\begin{align*}
f \approx e^{-\beta H}.
\end{align*}

\section{The BBGKY hierarchy}
So far, the transition to probability description did not simplified the problem much -- we are still dealing with the function of $N \approx 10^{23}$ variables.

We need to limit our ambitions and we will focus on the one-particle distribution, defined by
\begin{align*}
f_1(\bm{x},\bm{u},t) = N \int \prod_{i=2}^N \dd \, x_i \dd \, u_i f(\bm{x},\bm{u},t).
\end{align*}
The choice of index $1$ was arbitrary, because all the particles are identical. This is also the reason why $f_1$ is normalized to $N$:
\begin{align*}
\int f_1(\bm{x},\bm{u},t) \dd \bm{x} \dd \bm{u} = N
\end{align*}

For many purposes, the function $f_1$ captures all we need to know.

The average density of particles in the real space is
\begin{align*}
n(\bm{x},t) = \int f_1(\bm{x},\bm{u},t) \dd \bm{x},
\end{align*}
the average velocity of the particle is
\begin{align*}
n(\bm{x},t) = \int \bm{u} f_1(\bm{x},\bm{u},t) \dd \bm{u},
\end{align*}
and the energy flux is
\begin{align*}
\varepsilon(\bm{x},t) = \int \bm{u} E(\bm{u}) f_1(\bm{x},\bm{u},t) \dd \bm{u}.
\end{align*}

To derive the equation governing the evolution of $f_1$, let us see how it varies in time
\begin{align*}
\frac{\pd f_1}{\pd t} = N \int \prod_{i=2}^N \dd \, x_i \dd \, u_i \frac{\pd f}{\pd t} = N \int \prod_{i=2}^N \dd \, x_i \dd \, u_i \big\{ H, \, f \big\}.
\end{align*}
After some non-trivial arrangements we get
\begin{align} \label{boltzman}
\frac{\pd f_1}{\pd t} = \big\{ H_1, f_1 \big\} + \Big(\frac{\pd f_1}{\pd t} \Big)_{coll},
\end{align}
where
\begin{align*}
H_1 = \frac{p^2}{2m} + V(\bm{x}).
\end{align*}
We see that the evolution of $f_1$ is governed by the Liouville's equation plus the extra term, the \textit{collision integral}
\begin{align*}
\Big(\frac{\pd f_1}{\pd t} \Big)_{coll} = N(N-1) \int \dd \bm{x_2} \dd \bm{u_2} 
\frac{\pd U(\bm{x} - \bm{x_2})}{\pd {\bm{x}}} . \frac{\pd}{\pd \bm{u}} \int \prod_{i=3}^N \dd \bm{x_i} \dd\bm{u_i} f(\bm{x},\bm{u},t).
\end{align*}

Not surprisingly, the collisions are not governed by one-particle distribution, since it does not contain the relation of one particle to all others. Some of that information is carried by the \textit{two-particle distribution}
\begin{align*}
f_2(\bm{x},\bm{u},t) := N(N-1)\int \prod_{i=3}^N f(\bm{x},\bm{u},t) \dd \bm{x_i} \dd \bm{u_i}.
\end{align*}
Using the two-particle distribution, the collision integral can be written as
\begin{align*}
\Big(\frac{\pd f_1}{\pd t} \Big)_{coll} = \int \dd \bm{x_2} \dd \bm{u_2} \frac{\pd U(\bm{x} - \bm{x_2})}{\pd \bm{x}} . \frac{\pd f_2}{\pd\bm{u}}.
\end{align*}

To conclude, if we want to follow the evolution of one-particle distribution $f_1$, we also need to know something about the two-particle distribution. And again, to know the evolution of the two-particle distribution, we repeat the same procedure as before and find the Liouville-like equation for $f_2$ with additional term containing the three-particle distribution.

In general, the evolution of \textit{n-particle distribution} is described by
\begin{align} \label{bgky}
\frac{\pd f_n}{\pd t} = \big\{ H_n, f_n \big\} + \sum_{i=1}^n \int \dd \bm{u_{n+1}} \dd \bm{x_{n+1}} \frac{\pd U(\bm{x_i} - \bm{x_{n+1}}}{\pd \bm{x_i}}.\frac{\pd f_{n+1}}{\pd \bm{u_i}}.
\end{align}
The Hamiltonian 
\begin{align*}
H_n = \sum_{i=1}^n \Big(\frac{\bm{u_i}}{2m} + V(\bm{u_i})\Big) + \sum_{i<j\leq n} U(\bm{x_i} - \bm{x_j})
\end{align*}
includes external forces $V$ and interactions between the $n$ particles.

The set of equations \ref{bgky} is called the \textit{BBGKY hierarchy}.
So instead of the Liouville equation governing the function of $N \approx 10^{23}$ variables, we have the set of $10^{23}$ functions. 

But it is really more advantageous to work with the hierarchy of equations \ref{bgky}, as it is ready to be approximated. Depending on the particular problem at hands, we decide which terms are important and which terms can be ignored, thus truncating the problem to something useful.

The most simple and useful of these truncations is the Boltzmann equation -- the closed equation for the $f_1$ alone.

We already saw its general form
\begin{align*}
\frac{\pd f_1}{\pd t} = \big\{ H_1, f_1 \big\} + \Big(\frac{\pd f_1}{\pd t} \Big)_{coll},
\end{align*}
but the collision integral was dependent on $f_2$.

To proceed, let us define the two time scales:
the time between particle collisions $\tau$, and the collision time $\tau_{coll}$, which is the time that the process of collision lasts. In dilute gas, we have
\begin{align*}
\tau >> \tau_{coll},
\end{align*}
so most of the time, the $f_1$ follows the Hamiltonina evolution, with occasional, but abrupt perturbatons by the collision, that reflects the collision integral.
%This force K will be neglected in the subsequent text, because external force is not supposed to influence dynamics of collisions.
%
%The right hand side of Boltzmann equation, the $Q(f,f)$, is the collision integral.
%It is quite complicated expression
%and it is the major difficulty in dealing with Boltzmann equation.
%
%But we do not need it that precisely, as two-particle collisions do not significantly influence experimentally measured quantities.
%Therefore, it is often approximated.
%
%Before we show the standard way to approximate it (so called BGK approximation), we need to present the properties of $Q(f,f)$ that need to be preserved.
%
%\section{Collision invariants}
%Toto podla tonga.
%Collision invariants are the five functions that are orthogonal on $Q(f,f)$. In other words
%\begin{align} \label{invariance}
%\int Q(f,f) \psi_k(\bm{v}) d^3v = 0.
%\end{align} 
%These collision invariants read
%\begin{align} \label{invariants}
%\begin{split}
%\psi_0 &= 1, \\
%\psi_1 &= v_1, \\
%\psi_2 &= v_2, \\
%\psi_3 &= v_3, \\
%\psi_4 &= v^2.
%\end{split}
%\end{align}
%
%Because of linearity of scalar product \ref{invariance}, any function 
%\begin{align}
%\psi = a + \bm{b}.\bm{v} + c \, v^2
%\end{align}
%is the collision invariant of $Q(f,f)$.
%
%Interestingly, function 
%\begin{align} \label{nulling}
%\phi = \exp (a + \bm{b}. \bm{v} + c \, v^2)
%\end{align}
%is the solution of
%\begin{align}
%Q(f,f) = 0,
%\end{align}
%if $c$ is negative. Notice, that famous Maxwell function (or Maxwell distribution, or simply Maxwellian) is the spacial kind of such function \ref{nulling} 
%\begin{equation} \label{Maxwell}
%f^M(\bm{x,v,}t) = n(\frac{m}{2 \pi k_B T})^{3/2} e^{-\frac{m}{2k_B T}(\bm{v-u})^2}.
%\end{equation}
%
%Now we are ready to state what is BGK approximation of collision integral $Q(f,f)$.
%\section{BGK approximation}
%Instead of $Q(f,f)$, that is so complicated we do not even state it, we will use simpler operator 
%\begin{align}
%J(f) = w[f^M(\bm{x,v}) - f(\bm{x,v}]
%p\end{align}
%
%We can easily see that this approximation have two properties:
%\begin{enumerate}
%\item It expresses the tendency of $f$ towards Maxwellian distribution \ref{Maxwell} (because the time derivative of $f$ has sign towards $f^M$).
%\item It conserves the collision invariants \ref{invariants}
%\begin{align}
%\int \psi_k J(f) d^3 x d^3 v = 0
%\end{align}
%\end{enumerate}
%
%
%\section{Chapman-Enskog series}
%We are already familiar with this kind of expansion, now we will show it in the general context.
%In some respect this expansion is quite peculiar and deserves few comments before we suck the life out of it.
%
%\bigskip
%
%The distribution function f, that is explicitly time-dependent, will be expanded in the series of $f^{(n)}$, that have only implicit time-dependence. However, gas relaxes very fast, in time of order $10^{-11}$ seconds, towards time-independent distributions, hence this expansion is generally valid.
%
%
%The expansion parameter $\epsilon$ depended on the peculiarities of our model in previous chapters. 
%In this section, we will take $\epsilon$ equal to Knudsen number
%\begin{align}
%K_n = \frac{\lambda}{L},
%\end{align}
%that is ratio of $\lambda$, the mean free path of particle (the mean distance between two successive collisions) and spatial characteristic scale $L$ (for example, size of the obstacle in the flow, diameter of the flow etc.).
%
%\bigskip
%
%And the most interesting point comes last - the Chapman-Enskog series diverges.
%
%We are interested only in first two orders of the expansion, but higher order-approximation that lead to Burnett equations are problematic and were never applied systematically (for further details see Cercignani 1988 and 1990).
%
%
%\section{Conservation laws}
%What happens if we multiply Boltzmann equation with the collision invariants \ref{invariants} and integrate it?
%
%Since the collision integral in the right-hand side gets killed-off (by definition of invariants), 
%the equation reads
%\begin{equation}
%\int \psi_k(\partial_t + v_{\alpha}\partial_{\alpha}) f(\bm{x},\bm{v},t) d^3v = 0.
%\end{equation}
%Substituting $\psi_0 = 1$, $(\psi_1,\psi_2,\psi_3) = \bm{v}$ and $\psi_4 = v^2$, we get the general form of conservation laws:
%
%\begin{align}
%\begin{split}
%\partial_t\rho + \partial_{\alpha}(\rho u_{\alpha}) &= 0 \\
%\rho \partial_t u_{\alpha} + \rho u_{\beta}\partial_{\beta}u_{\alpha} &= - \partial_{\alpha}\hat{P}_{\alpha\beta} \\
%\rho \partial_t \Theta + \rho u_{\beta}\partial_{\beta}\Theta &= -\frac{2}{3} \partial_{\alpha}q_{\alpha} - \frac{2}{3} \hat{P}_{\alpha\beta} \Lambda_{\alpha\beta}
%\end{split}
%\end{align}
%where
%\begin{align}
%\begin{split}
%n(\bm{x}, t) &= \int f(\bm{x},\bm{v},t) \, d^3 v  \\
%p(\bm{x},t) &= m \, n(\bm{x},t) \\
%\rho u_{\alpha}(\bm{x},t) &= m \int v_{\alpha} f(\bm{x},\bm{v},t) \, d^3 v \\
%\Theta(\bm{x},t) &= k_B \, T(\bm{x},t) = \frac{m}{3n} \int (v_{\alpha} - u_{\alpha})(v_{\alpha} - u_{\alpha}) \, f(\bm{x},\bm{v},t) \, d^3 v \\
%\Lambda_{\alpha\beta} &= \frac{m}{2} (\pd_{\beta}u_{\alpha} + \pd_{\alpha} u_{\beta}) \\
%\hat{P}_{\alpha\beta} &= m \int (v_{\alpha} - u_{\alpha})(v_{\beta} - u_{\beta}) \, f(\bm{x}, \bm{v}, t) \, d^3v \\
%q_{\alpha}(\bm{x},t) &= \frac{m^2}{2} \int (v_{\alpha} - u_{\alpha})(v_{\beta} - u_{\beta}) (v_{\beta} - u_{\beta}) \, f(\bm{x}, \bm{v}, t) \, d^3v
%\end{split}
%\end{align}
%
%\section{Euler equation}
%
%$f^M$ into conservation laws:
%$\partial_t \rho + \partial_{\alpha}(\rho u_{\alpha}) = 0$ ... continuity equation\\
%$\rho \partial_t u_{\alpha} + \rho u_{\beta}\partial_{\beta}u_{\alpha} = - \partial_{\alpha}P$ ... Euler equation,\\
%where $P = n\Theta = n k_B T$\\
%$\rho \partial_t \Theta + \rho u_{\beta}\partial_{\beta}\Theta = -\frac{1}{c_V} \Theta \partial_{\alpha} u_{\alpha}$\\
%
%Chapman-Enskog:\\
%$f = f^0 + \epsilon f^1 + \epsilon^2 f^2 + ... = $
%
%It is special in every sense of word, we will introduce some of its useful properties:
