\chapter{Lattice gass cellular automata}
The first lattice gass cellular automaton was proposed in 1973 by Hardy, Pomeau and de Pazis, and it is called -- the HPP model.

Unfortunately, it could not do its job sufficiently well. For the reasons that we will sketch in this chapter, its macroscopic limit does not converge to Navier-Stokes equations.

\bigskip

In the subsequent chapters we will explore two different approaches how to make functional LGCA. They both build on the ideas behind this imperfect HPP.

Therefore, we will explain the basic principles of HPP first, and later on, we will upgrade it - either to FHP, Pair-interaction or their multi-dimensional variants.

\section{From CA to LGCA}

Lattice of HPP is the simple rectangular 2D grid. At every point of a grid, there is a node sitting in, and this node is composed of 4 cells, see Figure~\ref{rectangular}.

\begin{figure}[htbp]
 \centering
 \includegraphics[width=0.6\textwidth]{./img/hppnode}
 \label{rectangular}
 \caption{Rectangular grid}
\end{figure}

Each cell of the node can be in two states -- empty (white square) or occupied by the particle (yellow square).
The particle in this cell is heading to the diagonal node along the corresponding lattice vector.

\section{Update rule}
%The update rule should be design in such a manner, that is conserve physically relevant quantities, namely mass and momentum.

So far we have the grid full of particles.
For all LGCA models, update is done in two subsequent steps -- collision and propagation.

In the collision step, particles are swapped inside the node respecting two constraints - conservation of mass and conservation of momentum in the node.

In HPP, there are only two collision configurations, and one collision rule that guides them, see Figure~\ref{hpp-colision}.

\begin{figure}[H]
 \centering
 \includegraphics[width=0.7\textwidth]{./img/hpp_col}
 \label{hpp-colision}
 \caption{HPP colisions}
\end{figure}

These collision configurations are symmetric -- the first configuration is resolved to the other and vice versa.

It is easy to understand that there are no other collision configurations and collision rules. If any other state gets changed, it would break the conservation of momentum and would be physically unrealistic.

\section{Propagation:}

After the collision is resolved, propagation follows -- figure \ref{hpp-colision}.
During propagation, the particle moves from the nodes to nodes, along the lattice vectors that corresponds to the cells they occupied.
\begin{figure} [H]
 \centering
 \includegraphics[width=0.6\textwidth]{./img/HPPprop}
 \label{hpp-colision}
 \caption{Propagation of particle from upper-left cell}
\end{figure}

\bigskip

\section{Conservation laws}

We already noted that collision and propagation conserve momentum, and they obviously conserve mass, since particles are neither created, nor anihilated during update. 
Let us inspect these conservation laws in the more depth by considering symmetries of this model.

Suppose we are using periodic boundary condition. Then, the finite rectangular grid of HPP is actually a torus. It is easy to imagine that if we shift the grid by the lattice vector, we get the very same torus.
So this rectangular grid of HPP is symmetric with respect to translation.
As we know by Norther's theorems, translational symmetry implies conservation of momentum.

\bigskip

Also, the grid possesses a rough rotational symmetry - rotating the grid by 90 degrees, we get the same grid.

However, this rotation by 90 degrees is too crude.
In the following chapter, when we derive the hydrodynamic equations for this type of LGCA, we could see that four lattice vectors with rectangular symmetry leads to unrealistic equations, comparing to the better models.

\bigskip
Another implication of insufficient rotational symmetry are the non-physical quantities, that are conserved nevertheless -- so called spurious invariant \footnote{The later LGCAs also posses the spurious invariants - so-called Zanetti's invariants, but they are under level of noise, due to higher rotational symmetry or additional degrees of freedom in the node}.

Consider orientation of particles in the figure \ref{hpp-colision} before the collision.
One particle heads to north-east, the other south-west.
After the collision, one particle heads to north-west, other particle to south-east.
So the number of particles heading to the south, east, north and west do not change by collision (and neither by propagation).

To assert it more rigorously, let us decompose the total momentum into the cardinal directions:
\begin{align} 
P = P_N + P_S + P_E + P_W.
\end{align}
This total momentum $P$ is correctly conserved by HPP, but also quantities
\begin{align} \label{zanet}
P_{spur1} = P_N + P_E - P_S - P_W
\end{align}
and
\begin{align}
P_{spur2} = P_N + P_W - P_S - P_E
\end{align}
are conserved, although these quantities have no physical counterparts.

\bigskip

To conclude this chapter and finish-off the HPP, it is physically implausible because
\begin{enumerate}
\item angular momentum is not conserved due to insufficient rotational symmetry,
\item other non-physical quantities, so called \textit{spurious invariants} are conserved.
\end{enumerate}

Although it is a flawed model, it sparked interest of the wider community and various successful LGCAs evolved from HPP. In the next chapter, we will introduce the first physically relevant branch of LGCAs -- the FHP model.