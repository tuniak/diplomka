\chapter*{How to use this diploma thesis}

For the beginner in this field, we recommend to follow sequence of the chapters, since the theoretical part constitutes tutorial to lattice-gas cellular automata, but an experienced user can feel free to chose the topic he finds interesting.

\bigskip

In chapter one, we introduce the notion of cellular automata.

In chapter two, we continue with the special type cellular automata, the lattice-gas CA, that represents the original approach in CFD.

In chapter three, we present the better-known branch of LCCA, that started with FHP in two-dimensions and FCHC in three-dimensions.

In chapter four, we inspected microdynamics of FHP more theoretically and generally, so that the obtained results are valid for N-dimensional FHP-like LGCA, namely FCHC.

In chapter five, we derive macroscopic equations for FHP and FCHC.

In chapter six, we analyzed FCHC lattice and sketched FCHC collision algorithm as proposed by Henon.

In chapter seven, we introduced another successful branch of LGCA, the Pair-Interaction models. They are more comfortable to implement then FCHC in 3D, they are great model for ideal fluids, but they have some drawbacks for viscous fluids (viscosity is anisotropic).

Chapter eight concludes theoretical analysis of LGCA by showing what statistical properties does physical fluid exhibit, and that if we artificially impose its statistical properties on the LGCA, we obtain physically realistic model by every means, although this new generation of LGCA are beyond the scope of this thesis.

In chapter ten we introduce probabilistic methods that we intend to use in practical part, to inspect properties of fully-developed turbulence simulated by our models.

\bigskip

Practical part starts with the chapter ten, that summarizes what its content.

Chapter eleven contains general comments on the implementation.

Chapter twelve and thirteen discuss implementation of FCHC and standard algorithm of Pair-Interaction

Chapter fourteen motivates non-deterministic variant of Pair-interaction automata.

In chapter fifteen, we present results of the flow around obstacles simulated on our models.

Chapter sixteen is scientifically most challenging part, and although the obtained results are questionable from various positions, they constitute basis for the further research.


\addcontentsline{toc}{chapter}{Introduction}

