\chapter{FHP}
Again, this enhanced lattice gas cellular automaton is named after its inventors -- Frisch, Humpfrey and Pomeu. 
They proposed it in 1986 together with its three dimensional variant - the FCHC. 
%Comparing it to HPP, we have a lattice with better rotational symmetry and nodes with richer  set of collisions.
%
%Now that we presented the setting of FHP in two dimensions rather intuitively, let us explore the microdynamics of FHP in a more formal way. The formal treatment provides that the hydrodynamic equations that we derive are valid for FHP in arbitrary dimension (although finding an appropriate multidimensional lattice is non-trivial and possible only in 4 dimensions, as we will see).
In following section, we will graphically explain the basic principles of FHP, the later sections will be more general and the obtained results will be valid for arbitrary dimensional FHP-like automaton (most importantly FCHC).

\section{The lattice of FHP}
The whole improvement of FHP lies in a simple change - instead of the rectangular grid, FHP builds on a hexagonal grid. 
%The two dimensional plane can be uniformly covered by squares, but also by hexagon, that . This is the main improvement of FHP, 
All other properties are implied by the increased symmetry of the lattice.

On the figure \ref{FHPgrid}, we have a part of the hexagonal grid, and from one of the nodes, six lattice vectors point to the neighboring nodes.
Let us denote the set of the lattice vectors by $c_i,~i=1,2,3,4,5,6$.


\begin{figure}[htbp] \label{FHPgrid}
 \centering
 \includegraphics[width=0.6\textwidth]{./img/fhp_desc}
\end{figure}

The node is a set of the six cells and each of the cells corresponds to one of the lattice vectors.
Let us denote the state of the node by $\bm{n} = (n_1,n_2,n_3,n_4,n_5,n_6)$, where $n_i = 0$ stands for an empty $i^{th}$ cell, and $n_i = 1$ implies that there is a particle in the $i^{th}$ cell. 

State of a node with the position $\bf{r}$ on lattice will be denoted by $\bf{n(r)}$, whereas the state of the \textit{whole lattice} will be denoted by \textbf{n(.)}.

\section{Update rule}
As we know, the update happens in discrete time steps ($t=1,2,3...$) and consists of two subsequent steps - collision and propagation. Both of these steps are local, so they can be treated node by node.

\subsection{Propagation}

Propagation is the straight-forward phase and can be captured by the simple equation
\begin{align*}
S n_i(\bm{r}) = n_i(\bm{r} + \bm{c_i}). 
\end{align*}
%This equation means that state of the $i^{th}$ cell in node \textbf{n(r)} \textit{propagates} along the lattice vector $c_i$ to the neighboring node $\bf{n(r + c_i)}$, the figure \ref{FHPprop}.
If the cell is occupied by a particle, that means $n_i(\bm{r})=1$, it propagates along the corresponding lattice vector to the neighboring node, see the figure \ref{FHPprop}.

\begin{figure}[H]
 \centering
 \includegraphics[width=0.7\textwidth]{./img/FHPprop}
 \caption{FHP collisions without rest particle.}
 \label{FHPprop}
\end{figure}
The propagation, however, is preceded by the more difficult step -- the collision.

\subsection{Collision}

The purpose of the collision is to swap as many particles in the node as possible.
The only constraint on the collision rule is to preserve the number of particles (conservation of the mass) and to preserve the total momentum in the node.

These requirements lead only to a handful of collision configurations, see figure \ref{FHPcol}. For the simplicity, we are considering the FHP-I model, that does not included the "rest particles", otherwise there would be few more rules to add.

\begin{figure}[H]
 \centering
 \includegraphics[width=0.7\textwidth]{./img/FHPcol}
 \caption{FHP collisions without rest particle.}
 \label{FHPcol}
\end{figure}

We see that two and four particle configurations can be resolved in two different configurations. The resulting state need to be chosen randomly, with probability $1/2$ for each state to preserve the parity symmetry of the model. If we were systematically preferring one of them, we would introduce additional, non-physical invariant - chirality. Hence, we need to introduce non-determinism to the model.


We will express the probabilities of transition from state $n$ to state $n'$ by the probability matrix
\begin{align*}
A(n \rightarrow n') \geq 0.
\end{align*}
As we have 64 possible states of the node, matrix A is of dimension $64\times 64$.
For example, the submatrix of A that governs the head-on collisions is
\[
 A'=
  \begin{bmatrix}
    0 & \frac{1}{2} & \frac{1}{2} \\
    \frac{1}{2} & 0 & \frac{1}{2} \\
    \frac{1}{2} & \frac{1}{2} & 0 \\
  \end{bmatrix}
\]
Since the collisions are symmetric, matrix A is symmetric as well.

Also, collisions are invariant to rotations and reflections of the node
\begin{align*}
A(g(\bm{n}) \rightarrow g(\bm{n'})) = A(\bm{n} \rightarrow \bm{n'}),
\end{align*}
where $g \in G$, and G is the symmetry group of the node.
The normalization of probability leads to the semi-detailed balance
\begin{align} \label{smd}
\sum_{\bm{n}} A(\bm{n} \rightarrow \bm{n'}) = 1	
\end{align}

\section{Collision operator}
Interestingly, we can express the whole update step in one simple equation using collision operator $\Delta_i$:
\begin{align} \label{withcol}
n_i(t+1,r+c_i) = n_i(t,r) + \Delta_i(t,r)
\end{align}
where $\Delta_i$ is
\begin{enumerate}
\item $\Delta_i = 0$ if no collision is happening in $n(t,r)$. Then state of the cell $n_i(t,r)$ only propagates to $n_i(t+1,r+c_i)$.
\item $\Delta_i = 1$ if there is not particle in $n_i(t,r)$ yet, but gets there after collision. 
\item $\Delta_i = -1$ if there is particle in the $n_i(t,r)$, but after the collision, cell gets empty.
\end{enumerate}

\bigskip

%For example, $\Delta_i$ acquire very simple form for HPP
%\begin{equation*}
%\Delta_i = n_{i+1} n_{i+3}( 1 - n_i)(1 - n_{i+2}) - n_i n_{i+2}(1-n_{i+1})(1 - n_{i+3}).
%\end{equation*}
%As there are only 2 collision configuration in HPP, $\Delta_i$ has 2 terms.

By this reasoning, we obtain the form of collision operator for FHP-I model
\begin{align} \label{colop}
\begin{split}
\Delta_i = n_{i+1}n_{i+3}n_{i+5}(1-n_i)(1-n_{i+2})(1-n_{i+4})\\
-n_in_{i+2}n_{i+4}(1-n_{i+1})(1-n_{i+3})(1-n_{i+5})\\
 + \xi n_{i+1}n_{i+4}(1-n_i)(1-n_{i+2})(1-n_{i+3})(1-n_{i+5})\\
 +(1-\xi)n_{i+2}n_{i+5}(1-n_i)(1-n_{i+1})(1-n_{i+3})(1-n_{i+4})\\
 -n_in_{i+3}(1-n_{i+1})(1-n_{i+2})(1-n_{i+4})(1-n_{i+5}).
\end{split}
\end{align}

\section{Microscopic conservation laws}
We can easily prove that
%By substituting \ref{colop} for the collision operator, we can prove that
\begin{align*}
\begin{split}
\sum_i \Delta_i(t,r) &= 0,\\
\sum_i c_i \Delta_i(t,r) &= 0.
\end{split}
\end{align*}
by substituting \ref{colop} into these equations. Combining with \ref{withcol}, these equations imply the conservation of mass and momentum
%Therefore, the laws of mass and momentum conservation acquire the simple form
\begin{align} \label{cons1}
\begin{split}
\sum_i n_i(t+1, r + c_i) &= \sum_i n_i(t,r), \\
\sum_i c_i n_i(t+1, r + c_i) &= \sum_i c_i n_i(t,r).
\end{split}
\end{align}
By employing the aparatus of the statistical physics we will show that these microscopic conservation laws lead to physically plausible macroscopic description.

\section{Conservation of probabilities}
Let us define the phase space $\Gamma$ as the set of all possible states of the lattice $n(.)$.
Imagine we want to initialize cellular automaton with some macroscopic velocity $\bm{v_0}$, macroscopic pressure $p_0$, and macroscopic density $\rho_0$.
We can realize this macrostate by very many microstates of lattice $\bm{n(.)}$.
We assign initial probability to each of these microstates
\begin{align*}
P(0,\bm{n(.)}) \geq 0.
\end{align*}
As always, probabilities over whole lattice are normalized:
\begin{align*}
\sum_{\bm{n(.)}} P(0,\bm{n(.)}) = 1.
\end{align*}

In the statistical mechanics, Liouville's space state theorem postulates that the density of the phase space is constant. Microdynamics of our model implies equivalent theorem for LGCA:
\begin{align*}
P(t+1, \mathcal{E} \bm{n}(.)) = P(t, \bm{n}(.))
\end{align*}

It is obtained directly by applying the update formula
\begin{equation*}
\mathcal{E} \bm{n}(t,.) = \bm{n}(t+1,.)
\end{equation*}
However, for FHP and other non-deterministic models, the conservation of probability is governed by the more general Chapman-Kolmogorov equation
\begin{equation} \label{chapkol}
P(t+1,\mathcal{S} \bm{n'}(.)) = \sum_{\bm{n}(.) \in \Gamma} \prod_{\bm{r}} A(\bm{n}(\bm{r}) \rightarrow \bm{n'}(\bm{r})) P(t, \bm{n}(.)),
\end{equation}
that reduces to Liouville equations in deterministic case, where $A(\bm{n(r)} \rightarrow \bm{n'(r)})$ selects unique configuration.

\section{Mean occupation numbers}
Motivated by the ensemble formalism of statistical physics, we define mean occupation numbers
\begin{equation*}
N_i = \langle n_i \rangle = \sum_{n_i(.) \in \Gamma} n_i(.) P(t,n_i(.))
\end{equation*}
that naturally implies formula for mean mass density
\begin{equation} \label{ddens}
\rho(t,r) = \sum_i N_i(t,\bm{r})
\end{equation}
and the momentum density
\begin{equation} \label{mmom}
j(t,r) = \sum_i c_i N_i(t,\bm{r}).
\end{equation}

Due to conservation of probabilities, the conservation laws \ref{cons1} implies conservation of the mean populations
\begin{equation} \label{macro1}
\sum_i N_i(t+1,\bm{r+c_i}) = \sum_i N_i(t,\bm{r}), 
\end{equation}
\begin{equation} \label{macro2}
\sum_i c_i N_i(t+1,\bm{r+c_i}) = \sum_i c_i N_i(t,\bm{r}).
\end{equation}

\section{Equilibrium occupation numbers}
We are interested in the equilibrium solutions of the Chapman-Kolmogorov equation \ref{chapkol}. Because collisions on the lattice are purely local, we are looking for the solution of the form
\begin{align} \label{formula1}
P(\bm{n}(.)) = \prod_{\bm{r} \in \mathcal{L}} p(\bm{n}(\bm{r})).
\end{align}
Maximization of the entropy suggest $p(n(.))$ of the form
\begin{align} \label{formula2}
p(\bm{n}(\bm{r})) = \prod_i N_i^{n_i} (1 - N_i)^{(1-n_i)}.
\end{align}
where $N_i^{n_i} (1 - N_i)^{(1-n_i)}$ is probability that mean occupation number is $N_i$.

Substituting the formula \ref{formula1} and \ref{formula2} into Chapman-Kolmogorov equation \ref{chapkol} gives
\begin{align} \label{rovnicka}
N_i^{n'_i} (1 - N_i)^{(1-n'_i)} = \sum_{\bm{n} \in \Gamma} \prod_{\bm{r}} A(\bm{n} \rightarrow \bm{n'}) N_i^{n_i} (1 - N_i)^{(1-n_i)},
\end{align}
for each state $\bm{n}$ of $64$ possible states (or $2^b$ for automaton with $b$ lattice vectors). So in fact the last equation is a set of $64$ equations.

It is a non-trivial fact, that this set has a solution, and the following lemma states, that it is given by Fermi-Dirac distribution. The theorem is stated for $D$ dimensional cellular automaton with $b$ lattice vectors $c_i,~i=1,2...b$.

\textbf{Theorem 1:}
The following statements are equivalent:
\begin{enumerate}
\item $N_i^{eq}$ are solutions of \ref{rovnicka}.\\
\item $N_i^{eq}$ are solutions of the set of b equations:
\begin{equation}
\Delta_i(N) = \sum_{nn'}(n'_i - n_i)A(n \rightarrow n')\prod_j N_j^{n_j}(1-N_j)^{1-n_j}
\end{equation} 

\item $N_i^{eq}$ are given by Fermi-Dirac distribution
\begin{align} \label{fd}
N_i^{eq} = \frac{1}{1 + \exp(h + \bm{q}.\bm{c_i})},
\end{align}
where h is real number and \textbf{q} is D-dimensional vector.
\end{enumerate}

The proof of this theorem can be found in the Appendix C of \cite{frisch}.
Further, non-deterministic collisions and semi-detailed balance \ref{smd} lead to the \textit{Universality theorem}, stating that these automata admit universal equilibrium solutions, completly factorized over all nodes and cells, with mean populations $N_i$ given by the Fermi-Dirac distribution, dependent only on the density $\rho$ and mass current $j = \rho \bm{u}$ and independent of the transition probabilities $A(\bm{n} \rightarrow \bm{n'})$ \cite{frisch}.

The idea of the proof is as follows:

Unknown parameters $h$ and $\bm{q}$ can be calculated in terms of mass density and velocity by inversion of the relations
\begin{equation}
\rho = \sum_i N_i = \sum_i \frac{1}{1+ exp(h + \bm{q}.\bm{c_i})}
\end{equation}
\begin{equation}
\bm{u} = \sum_i \bm{c_i} \bm{N_i} = \sum_i \frac{\bm{c_i}}{1+ exp(h + \bm{q}.\bm{c_i})}
\end{equation}
For FHP, explicit solutions are available only in few special cases \cite{frisch}.
It is not surprising to obtain Fermi-Dirac distribution in equilibrium, due to the exclusion principle (cells are empty or occupied by one particle).

To proceed, we will use perturbation of equilibrium solutions $N_i^{eq}$ in powers of $\bm{u}$ for small Mach numbers ($\bm{u}/c_{\mathrm{sound}}$). 
Up to the second order\footnote{The expansio to the second order is required, because the advection in Navier-Stokes equations will emerge from the quadratic term}, the expansion reads \cite{frisch}
\begin{equation} \label{eou}
N_i^{eq}(\rho,\bm{u}) = \frac{\rho}{b} + \frac{D\rho}{c^2 b}\bm{c_i}.\bm{u} = \rho G(\rho) Q_{i\alpha\beta}u_{\alpha}u_{\beta} + O(u^3)
\end{equation}
where 
\begin{align}
Q_{i\alpha\beta} = c_{i\alpha} c_{i\beta} - \frac{c^2}{D} \delta_{\alpha\beta}
\end{align}
and
\begin{align}
G(\rho) = \frac{D^2}{2c^4b}\frac{b-2\rho}{b-\rho}.
\end{align}
We see that for $\rho= b/2$, the coefficient $G(\rho)$ vanishes.
%Realizing that
%\begin{equation}
%\sum_i c_{i\alpha} c_{i\beta} = 
%\end{equation}

\section{Chapman-Enskog expansion}
Because relaxation towards equilibrium values happens in a few updates of the automaton, it is standard procedure to expand the occupation numbers $N_i(t,r)$ around the equilibrium occupation numbers $N_i^{eq}(\rho,\bm{u})$:
\begin{equation} \label{chap}
N_i(t,r) = N_i^0(t,r) + \epsilon N_i^1(t,r) + \mathcal{O}(\epsilon^2) 
\end{equation} 
%Following formulas
%\begin{equation}
%\rho = \sum_i N_i(t,r) = \sum_i N_i^{eq}(\rho,\bm{u})
%\end{equation}
%\begin{equation}
%\bm{j} = \sum_i c_i N_i(t,r) = \sum_i c_i N_i^{eq}(\rho,\bm{u})
%\end{equation}
%imply that


The equations of mass and momentum conservation \ref{macro1} and \ref{macro2} can be equivalently stated in the form
\begin{equation} \label{macro_m}
\sum_i N_i(t+1,r+c_i) - N_i(t,r) = 0 ,
\end{equation}
\begin{equation} \label{macro_p}
\sum_i c_i (N_i(t+1,r+c_i) - N_i(t,r)) = 0,
\end{equation}
that is more suitable for our purpose.

Expansion of $N_i(t+1,r+c_i)$ around $N_i(t,r)$ leads to
\begin{equation} \label{rozvoj t+1}
\begin{split}
N_i(t+1,r+c_i) = N_i(t,r) + \partial_t N_i(t,r) + c_{i\alpha} \partial_{\alpha} N_i(t,r) \\ 
+ \frac{1}{2} \partial_t \partial_t N_i(t,r) + \frac{1}{2} c_{i\alpha}c_{i\beta} \partial_{\alpha} \partial_{\beta} N_i(t,r) + c_{i\alpha} \partial_t \partial_{\alpha} N_i(t,r) + \mathcal{O}(\partial^3).
\end{split}
\end{equation}
\bigskip
The expression above is the mixture of various physical phenomena -- diffusion, advection, propagation of the sound waves or relaxation towards local equilibria. Each phenomena has its typical spatial and temporal scale, that corresponds to different powers of $\epsilon$, see the tables below. %\ref{scalings}.

\begin{center} 
    \begin{tabular}{| l | l | l | l |}
    \hline
    \multicolumn{3}{|c|}{ \label{scalings} TEMPORAL SCALES}\\ \hline
    \textbf{Scale} & \textbf{Rescaling of time} & \textbf{Phenomena} \\ \hline
    1 step & t & Relaxation towards local equilibrium \\ \hline
    100 steps & $t_1 = \frac{1}{100} t = \epsilon t$ & advection, sound waves (perturbation of mass and density) \\ \hline
    10 000 steps & $t_2 = \frac{1}{10000} t = \epsilon^2 t$ & diffusion \\ \hline
    \label{scalings}
    \end{tabular}
\end{center}


\begin{center}
    \begin{tabular}{| l | l | l | l |}
    \hline
    \multicolumn{3}{|c|}{SPATIAL SCALES}\\ \hline
    \textbf{Scale} & \textbf{Rescaling of length} & \textbf{Phenomena} \\ \hline
    1 lattice unit & \textbf{r} & Relaxation towards local equilibrium \\ \hline
    100 lattice units & $\bm{r_1} = \frac{1}{100} \bm{r} = \epsilon \bm{r}$ & diffusion, advection, sound waves\\ \hline
    \end{tabular}
\end{center}

To derive the hydrodynamical equation, we will exploit the multi-scale technique. It starts by grouping-up the terms of hydrodynamical temporal and spatial scales.


\bigskip
Using the rescaled length and time, we deduce the rescaled differential operators
\begin{equation} \label{oper}
\begin{split}
\partial_t = \epsilon \partial_t^{(1)} + \epsilon^2 \partial_t^{(2)} \\
\partial_{\alpha} = \epsilon \partial_{\alpha}^{(1)}
\end{split}
\end{equation}

Now, we are ready to expand the conservation laws \ref{macro_m} and \ref{macro_p} in Chapman-Enskog series. We insert expansion \ref{rozvoj t+1} of $N_i(t+1,r+c_i)$, then we insert expansion of $N_i(t,r)$ according to \ref{chap}. Finally, we substitute differential operators according to \ref{oper}.

From the expansion, we present only the terms of order $\epsilon$, that we are interested in.

From the conservation of mass \ref{macro_m} we get
\begin{equation}
\partial_t^{(1)} \sum_i N_i^{(0)} + \partial_{\beta} \sum_i c_{i\beta} N_i^{(0)} = 0,
\end{equation}
and from the conservation of momentum \ref{macro_p} we get
\begin{equation}
\partial_t^{(1)} \sum_i c_{i\alpha} N_i^{(0)} + \partial_{\beta} \sum_i c_{i\alpha} c_{i\beta} N_i^{(0)} = 0.
\end{equation}

Using definition of mass density, and the following definition of the \textit{momentum flux tensor}
\begin{equation}
P^{(0)}_{\alpha\beta} := \sum_i c_{i\alpha} c_{i\beta} N_i^{(0)} = \sum_i c_{i\alpha} c_{i\beta} N_i^{eq}(\rho, \bm{u}).
\end{equation}
we can write the conservation laws in the shorter form
\begin{equation} 
\begin{split}
\partial_t^{(1)} \rho + \nabla^{(1)}(\rho \bm{u}) = 0, \\ 
\partial_t^{(1)} (\rho u_{\alpha}) + \nabla^{(1)} P^{(0)}_{\alpha\beta} = 0 \label{eul_primitive}
\end{split}
\end{equation}

In the first equation, we can recognize the familiar continuity equation, but we have some work to do with the second one.

%Now is the time to observe, that:
%\begin{equation} \label{2mom}
%\sum_i c_{i\alpha}c_{i\beta} = 3\delta_{\alpha\beta}
%\end{equation}
%Using this observation and inserting formula for equilibrium occupation number \ref{eou} we can write:
%\begin{equation}
%P^{(0)}_{\alpha\beta} = \frac{\rho}{2} \delta_{\alpha\beta} 
%+ \rho G(\rho) \sum_i c_{i\alpha}c_{i\beta} Q_{i\gamma\delta} u_{\gamma} u_{\delta} + \mathcal{O}(u^4)
%\end{equation} 
Substituting \ref{eou} for $N_i^{eq}$ into momentum flux tensor, its components read (after some algebraic simplification)
\begin{equation} \label{FHPT}
\begin{split}
P_{xx}^{(0)} = \frac{\rho}{2}g(\rho)(u_x^2 - u_y^2) + \frac{\rho}{2},\\
P_{yy}^{(0)} = \frac{\rho}{2}g(\rho) (u_x^2 - u_y^2) + \frac{\rho}{2},\\
P_{xy}^{(0)} = P_{yx}^{(0)} = \rho g(\rho)u_xu_y,
\end{split}
\end{equation}
where we defined $g(\rho) =  \frac{3-\rho}{6 - \rho}$.

Unfortunately, it does not match the momentum flux tensor of Navier-Stokes equation
\begin{equation} \label{NST}
\begin{split}
P_{xx} = \rho \, u_x^2 + p\\
P_{yy} = \rho\, u_y^2 + p\\
P_{xy} = P_{yx} = \rho u_x u_y
\end{split}
\end{equation}

Let us examine why are these tensors different.

For low values of $\bm{u}^2$, pressure $p$ is given by isothermal relation $p = \frac{\rho}{2} = \rho c_s^2$, where $c_s = \frac{1}{\sqrt{2}}$ is the speed of sound\cite{wolf}.

What about $g(\rho)$?  

The disease of FHP, FCHC and all the lattice-gas cellular automata to follow is that $g(\rho)$ is never equal to 1, as it is in Navier-Stokes equations.

The reason lies in the broken \textit{Galilei invariance} in these automata as they all have discrete rotational symmetry (by $60 \degree$ in FHP and between $60 \degree$ and $45 \degree$ in FCHC).

%In the final chapter on theory of LGCA, we will show the fundamental treatment of this flaw. 
Symptomatic treatment, such as rescaling of time
\begin{align} \label{frac_resc}
t \rightarrow \frac{t}{g(\rho)},
\end{align}
does not solve all the associated problems (D'Humier et al. 1987).
This flaw requires fundamental treatment, and leads to the new generation of automata that we will present in the final chapter on LGCA theory.

However, using this rescaled time, and setting density to be constant ($\rho = \rho_0$) for all terms except the pressure, we get the FHP version of incompressible Euler equation
\begin{equation}
\frac{\partial \bm{u}}{\partial t} + (\bm{u} \bm{\nabla}) \bm{u} = -\bm{\nabla} P,
\end{equation}
but the pressure $P$ is still the function of density and velocity
\begin{equation} \label{press}
P = (\frac{\rho}{2\rho_0 g(\rho_0)} - \bm{u}^2).
\end{equation}

This is as far as we can get in the first approximation.
The derivation of the Navier-Stokes equations
\begin{align}
\begin{split}
\bm{\nabla . u} &= 0, \\
\pd_t \bm{u} + (\bm{u \nabla})\bm{u} &= - \bm{\nabla} P + \nu \, \nabla^2 \bm{u}
\end{split}
\end{align}
would require terms of the order $\epsilon^2$  from the Chapman-Enskog expansion to add in the equations \ref{macro_m} and \ref{macro_p}, as shown in \cite{frisch}.

%When the collision is resolved in every node, propagation follows.
%This phase is straigt-forward -- 
%
%\begin{figure}[H]
% \centering
% \includegraphics[width=0.7\textwidth]{./img/FHPprop}
% \caption{FHP collisions without rest particle.}
% \label{FHPprop}
%\end{figure}

%Therefore, every node consists of the six cells corresponding to these lattice vectors.
 
%Let us denote the state of the node by $\bm{n} = (n_1,n_2,n_3,n_4,n_5,n_6)$, where $n_i = 0$ stands form empty $i^{th}$ cell, and $n_i = 1$ implies that there is a particle in the $i^{th}$ cell.
% 
%We can actually identify lattice velocities of the particles with lattice vectors, as we are free to set their norm to be the same.
%
%\section{Update rule}
%In all LGCAs, the update of the lattice happens in discrete time steps and consists of two subsequent steps - collision and propagation. Both of these steps are local, so they can be treated node by node.
%
%and will be denoted by $t=1,2,3...$ and so forth. Update can be divided in two subsequent steps - collision and propagation. Both of these steps are local, so they can be treated node by node.

%The purpose of the collision is to swap as many particles in the node, as conservation of mass and momentum allows. This leads to the desirable minimization of viscosity.

%The only constraint on the collision rule is to preserve the number of particles (conservation of mass) and to preserve the total momentum in the node (conservation of momentum).

%
%Comparing it to HPP, we have a lattice with better rotational symmetry and nodes with richer  set of collisions.
%
%Now that we presented the setting of FHP in two dimensions rather intuitively, let us explore the microdynamics of FHP in a more formal way. The formal treatment provides that the hydrodynamic equations that we derive are valid for FHP in arbitrary dimension (although finding an appropriate multidimensional lattice is non-trivial and possible only in 4 dimensions, as we will see).
%
%\section{Convention that we will use}
%State of the node will be denoted by $\bm{n} = (n_1,n_2,n_3,n_4,n_5,n_6)$, where $n_i = 0$ stands for the empty $i^{th}$ cell, and $n_i = 1$ implies that there is a particle in the $i^{th}$ cell.
%
%State of a node with the position $\bf{r}$ on lattice will be denoted by $\bf{n(r)}$, whereas the state of the \textit{whole lattice} will be denoted by \textbf{n(.)}.
%
%As we know, the update happens in discrete time steps that we will denote by $t=1,2,3...$ and so forth.
%
%\section{Propagation}
%Propagation is the straight-forward step and can be captured by the simple equation
%\begin{align*}
%S n_i(\bf{r}) = n_i(\bf{r + c_i}). 
%\end{align*}
%This equation means that state of the $i^{th}$ cell in node \textbf{n(r)} \textit{propagates} along the lattice vector $c_i$ to the neighboring node $\bf{n(r + c_i)}$, the figure \ref{FHPprop}.
%
%\section{Collision}
%As we already sketched, the purpose of the collision is to swap as many particles in the node, as conservation of mass and momentum allows.
%
%For FHP in two dimensions, we have $2^6 = 64$ configurations, and whole bunch of collision configurations among them (figure \ref{FHPcol}).

%
%As we see, head-on 2-particle collisions and 4-particle collisions can result in two different states. If we wanted to preserve determinism, and we were systematically choosing only one of them, we would introduce additional, non-physical invariance - the model would become chiral. If we want to preserve the parity symmetry of the model, we need to assign equal probability to either of two final states.Hence, we are introducing non-determinism to the model.
%
%We will express the probabilities of transition from state $n$ to state $n'$ by probability matrix:
%\begin{equation}
%A(n \rightarrow n') \geq 0
%\end{equation}
%As we have 64 possible states of the node, matrix A is of dimension $64\times 64$.
%For example, the cell of matrix A that governs the head-on collisions looks like this:
%\[
% A'=
%  \begin{bmatrix}
%    0 & \frac{1}{2} & \frac{1}{2} \\
%    \frac{1}{2} & 0 & \frac{1}{2} \\
%    \frac{1}{2} & \frac{1}{2} & 0 \\
%  \end{bmatrix}
%\]
%Since the collisions are symmetric, matrix A is symmetric as well.
%
%Also, collisions are invariant to rotations or reflections of the node:
%\begin{equation}
%A(g(n) \rightarrow g(n')) = A(n \rightarrow n')
%\end{equation}
%where $g \in G$, and G is the symmetry group of the node.

%\section{Collision operator}
%Interestingly, we can express the whole update step in one simple equation using collision operator $\Delta_i$:
%\begin{equation}
%n_i(t+1,r+c_i) = n_i(t,r) + \Delta_i(t,r)
%\end{equation}
%If this equation works, then, $\Delta_i$ must be:
%\begin{enumerate}
%\item $\Delta_i = 0$ if no collision is happening in $n(t,r)$. Then state of the cell $n_i(t,r)$ only propagates to $n_i(t+1,r+c_i)$.
%\item $\Delta_i = 1$ if there is not particle in $n_i(t,r)$ yet, but gets there after collision. 
% \item $\Delta_i = -1$ if there is particle in the $n_i(t,r)$, but after collision, cell gets empty.
%\end{enumerate}
%
%\bigskip
%
%For example, $\Delta_i$ acquire very simple form for HPP:
%\begin{equation}
%\Delta_i = n_{i+1} n_{i+3}( 1 - n_i)(1 - n_{i+2}) - n_i n_{i+2}(1-n_{i+1})(1 - n_{i+3})
%\end{equation}
%As there are only 2 collision configuration in HPP, $\Delta_i$ has 2 terms.
%If either collision happens, corresponding term is 1.
%
%\section{Microscopic conservation laws}
%Using collision operator, conservation of mass and momentum is this simple:
%\begin{subequations}
%\begin{align}
%\sum_i \Delta_i(t,r) &= 0,\\
%%
%\sum_i c_i \Delta_i(t,r) &= 0
%\end{align}
%\end{subequations}
%(Prove: unfold $\Delta$s)
%Conservation laws can be equivalently expressed in the form
%\begin{align} \label{cons1}
%\begin{split}
%\sum_i n_i(t+1, r + c_i) &= \sum_i n_i(t,r), \\
%\sum_i c_i n_i(t+1, r + c_i) &= \sum_i c_i n_i(t,r).
%\end{split}
%\end{align}