\chapter{Statistical description of FHP}

\section{From microcosmos to macroworld}
So far, we formalized microdynamics of LGCA.
However, this microdynamics is physically unrealistic, this model is not even approximation of physically valid microdynamics.
However, in last section we proved, that our model obeys conservation of mass and momentum in the nodes.
By employing aparatus of statistical mechanics we will show that these microscopic conservation laws leads to macroscopic physically valid popis.
\\
To do that, we will employ aparatus of statistical mechanics.

\\
\section{Liouville's theorem a.k.a. conservation of probabilities}
We define phase space $\Gamma$ as set of all possible states of the lattice $n(.)$.\\
Imagine we want to initialize cellular automaton with some macroscopic velocity $\bm{v_0}$, macroscopic pressure $p_0$, and macroscopic density $\rho_0$.
We can realize this macrostate by very many microstates of lattice $n(.)$.
We assign initial probability to each of these microstates: $P(0,s(.)) \geq 0$. Of course, probabilities over whole lattice are normed to one:\\
$\sum_s(.) P(0,s(.)) = 1$.\\
\\
In statistical mechanics, Liouville's space state theorem holds valid.
It says that density of the phase space is constant.\\
Microdynamics of our model implies equivalent theorem for FHP:\\
$P(t+1, \mathcal{E} s(.)) = P(t, s(.))$\\
\\
It is standard procedure in statistical mechanics that instead of considering mean values over time intervals.



Lattice vectors $c_i,~i=1...6$ can also be interpreted as lattice velocities, if we consider that velocity of a particle is given by $\frac{c_i}{\Delta t}$, and $\Delta t = 1$, since time steps are discrete.
\\

We will try to keep formalism as simple as possible, but robust and general enough so the results will be portable to other types of cellular automata.\\
\\

The state of the whole node is determined by the occupation numbers $n_i\in {0,1},~i=1...6$. $n_i=1$ means there is an particle in the $i^{th}$ cell, $n_i = 0$ means the cell is empty. 

$c_i$ will denote corresponding lattice vector, and it denotes the path along which is particle from the cell $i$ transported to the neighbouring node.\\
\\
The nice thing about FHP is that all lattice vectors are of equal length, therefore it make sense to normalize them to 1.\\
Since the cellular automaton evolve in discrete time steps equal to 1,
we may think of lattice velocities as equivalent to lattice vectors.\\
\\
It is easy to show that the lattice vectors $c_i$ obey several useful formulas:\\
a) Their first moment is equal to zero due to symmetry of lattice:\\
$\sum_{i=1}^{6} c_i = 0$  \\
b) 




We see the distance of the cells is uniform over the lattice, 
so the absolute value of velocity will be same for all particles.
We may normalize it: $\mid{c_i}\mid = 1$ \\

And we naturally define coordinate representation of $c_i$ to be:\\

$ c_i = (cos(i\pi/3) , sin(i\pi/3))$ for $i = 1,...,6 $

The first three moments of lattice velocity $c_i$ read: \\ 
\\

$\sum_{i=1}^{6} c_{i\alpha}c_{i\beta} = \delta_{\alpha\beta}$ \\
$\sum_{i=1}^{6} c_{i\alpha}c_{i\beta}c_{i\gamma} \\ $

Streaming, or propagation, is defined as:\\
$ n_i(t + 1, r + c_i) = \textit{S}n_i(t, r) $

bude nasledovat definicia koliznej funkcie 