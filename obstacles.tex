\chapter{Simulation of the flow around the obstacles}

The microdynamics of the lattice-gas cellular automata is non-physical, and to obtain the physical velocity, we need to compute its mean value over thousands of nodes.
%
%In previous chapter, we explained microdynamics of FCHC and PI, that is non-physical.
%However, if we compute the mean velocity over thousands of nodes, we obtain physical velocity.

Functions that compute velocity are trivial and similar for Pair-interaction and FCHC alike, so we comment only on FCHC implementation.

\begin{lstlisting}
void compute_velocity(int***grid, double****v, int X, int Y, int Z, int side, int I, int J, int K)
{
	// The mean velocity is computed over 'side'*'side'*'side' nodes..
	double N = side*side*side;
	int i, j, k, l, m, n;
	int x, y, z;

#pragma omp parallel for private (i,j,k,l,m,n,x,y,z)
	for (i = 0; i < I; ++i)
	{
		for (j = 0; j < J; ++j)
		{
			for (k = 0; k < K; ++k)
			{
				// Null the velocities from previous time-step				
				v[i][j][k][0] = 0;
				v[i][j][k][1] = 0;
				v[i][j][k][2] = 0;
				
				for (x = i*side; x < (i + 1)*side; ++x)
				{
					for (y = j*side; y < (j + 1)*side; ++y)
					{
						for (z = k*side; z < (k + 1)*side; ++z)
						{
							n = grid[x][y][z];
							for (m = 0; m < 24; ++m)
							{
								if (n & C[m])
								{
									for (l = 0; l < 3; ++l)
									{
										v[i][j][k][l] += c[m][l];
									}
								}
							}
						}
					}
				}
				v[i][j][k][0] /= 0;
				v[i][j][k][1] /= 0;
				v[i][j][k][2] /= 0;
			}
		}
	}
}
\end{lstlisting}

In the next section, we will present the simulation of the flow around the spherical obstacle and round plate.
Particles with velocity in Z direction are flowing from the plain $z=1$, and they propagate freely out of the tunnel from $z = max - 1$. 

The flow is enclosed in the tunnel with the square intersection. 
The no-slip condition is imposed on the tunnel and obstacles alike, and it is guided by the Propagation function.
\begin{lstlisting}
void Propagation(int***from, int***to, int**table, int X, int Y, int Z)
{
....
....
....
....
						// If there is an obstacle in the node, particle's velocity is reversed.
						// e.g. particle with velocity [1,0,-1,0] gains velocity [-1,0,1,0] by the reflection from the obstacle
						// Hence, in the next step, it propagates to the node where it come from
						if (to[new_x][new_y][new_z] & OBS)
							to[new_x][new_y][new_z] |= Reverse[i];
						else
							to[new_x][new_y][new_z] |= C[i];
					}
....
....
....
....
}
\end{lstlisting}

\section{Flow around the sphere} 
Each velocity vector is computed as a mean over $N = 80^3 / 8 = 64000$ nodes for PI and $N = 80^3 = 521 000$ for FCHC.

\begin{figure}[H]
 \centering 
 \includegraphics[width=0.9\textwidth]{../PI_sphere_80/velocity_70}
 \label{transitions}
 \caption{PI - flow around sphere, time 70}
\end{figure}


\begin{figure}[H]
 \centering 
 \includegraphics[width=0.9\textwidth]{../PI_sphere_80/velocity_170}
 \label{transitions}
 \caption{PI - flow around sphere, time 170}
\end{figure}


\begin{figure}[H]
 \centering 
 \includegraphics[width=0.9\textwidth]{../PI_sphere_80/velocity_320}
 \label{transitions}
 \caption{PI - flow around sphere, time 320}
\end{figure}


\begin{figure}[H]
 \centering 
 \includegraphics[width=0.9\textwidth]{../PI_sphere_80/velocity_460}
 \label{transitions}
 \caption{PI - flow around sphere, time 460}
\end{figure}


\begin{figure}[H]
 \centering 
 \includegraphics[width=0.9\textwidth]{../PI_sphere_80/velocity_700}
 \label{transitions}
 \caption{PI - flow around sphere, time 700}
\end{figure}


\begin{figure}[H]
 \centering 
 \includegraphics[width=0.9\textwidth]{../PI_sphere_80/velocity_1000}
 \label{transitions}
 \caption{PI - flow around sphere, time 1000}
\end{figure}


\begin{figure}[H]
 \centering 
 \includegraphics[width=0.9\textwidth]{../PI_sphere_80/velocity_1500}
 \label{transitions}
 \caption{PI - flow around sphere, time 1500}
\end{figure}


\begin{figure}[H]
 \centering 
 \includegraphics[width=0.9\textwidth]{../PI_sphere_80/velocity_2300}
 \label{transitions}
 \caption{PI - flow around sphere, time 2300}
\end{figure}


\section{Flow around the disk}
Flow around the disk was performed on the Pair-interaction model only, the macroscopic velocity field was obtained by averaging over $N = 40^3 /8 = 8000$ nodes.

\begin{figure}[htbp]
 \centering 
 \includegraphics[width=0.9\textwidth]{../PI_plate_40/velocity_50}
 \label{transitions}
 \caption{PI - flow around plate, time 50}
\end{figure}

\begin{figure}[htbp]
 \centering 
 \includegraphics[width=0.9\textwidth]{../PI_plate_40/velocity_150}
 \label{transitions}
 \caption{PI - flow around plate, time 150}
\end{figure}

\begin{figure}[htbp]
 \centering 
 \includegraphics[width=0.9\textwidth]{../PI_plate_40/velocity_250}
 \label{transitions}
 \caption{PI - flow around plate, time 250}
\end{figure}


\begin{figure}[htbp]
 \centering 
 \includegraphics[width=0.9\textwidth]{../PI_plate_40/velocity_350}
 \label{transitions}
 \caption{PI - flow around plate, time 350}
\end{figure}

\begin{figure}[htbp]
 \centering 
 \includegraphics[width=0.9\textwidth]{../PI_plate_40/velocity_350}
 \label{transitions}
 \caption{PI - flow around plate, time 350}
\end{figure}

\begin{figure}[htbp]
 \centering 
 \includegraphics[width=0.9\textwidth]{../PI_plate_40/velocity_550}
 \label{transitions}
 \caption{PI - flow around plate, time 550}
\end{figure}

\begin{figure}[htbp]
 \centering 
 \includegraphics[width=0.9\textwidth]{../PI_plate_40/velocity_750}
 \label{transitions}
 \caption{PI - flow around plate, time 750}
\end{figure}

\begin{figure}[htbp]
 \centering 
 \includegraphics[width=0.9\textwidth]{../PI_plate_40/velocity_1000}
 \label{transitions}
 \caption{PI - flow around plate, time 1000}
\end{figure}

\begin{figure}[htbp]
 \centering 
 \includegraphics[width=0.9\textwidth]{../PI_plate_40/velocity_1500}
 \label{transitions}
 \caption{PI - flow around plate, time 1500}
\end{figure}

\begin{figure}[htbp]
 \centering 
 \includegraphics[width=0.9\textwidth]{../PI_plate_40/velocity_2000}
 \caption{PI - flow around plate, time 2000}
\end{figure}

\begin{figure}[htbp]
 \centering 
 \includegraphics[width=0.9\textwidth]{../PI_plate_40/velocity_2500}
 \caption{PI - flow around plate, time 2500}
\end{figure}
