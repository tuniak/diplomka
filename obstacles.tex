\chapter{Study of the flow around the obstacles}

In previous chapter, we explained microdynamics of FCHC and PI, that is non-physical.
However, if we avarage velocities of the particles over sufficiently large region, we obtain physical velocity field.


The function that implements computation of velocity from LGCA lattice is simple and similar for FCHC and PI, so we comment only on FCHC implementation.

\begin{lstlisting}
void compute_velocity(int***grid, double****v, int X, int Y, int Z, int side, int I, int J, int K)
{
	// We compute mean velocity over the cube with the side of length 'side'.
	// N is the number nodes in this cube.
	double N = side*side*side;
	int i, j, k, l, m, n;
	int x, y, z;

#pragma omp parallel for private (i,j,k,l,m,n,x,y,z)
	for (i = 0; i < I; ++i)
	{
		for (j = 0; j < J; ++j)
		{
			for (k = 0; k < K; ++k)
			{
				for (x = i*side; x < (i + 1)*side; ++x)
				{
					for (y = j*side; y < (j + 1)*side; ++y)
					{
						for (z = k*side; z < (k + 1)*side; ++z)
						{
							n = grid[x][y][z];
							for (m = 0; m < 24; ++m)
							{
								if (n & C[m])
								{
									for (l = 0; l < 3; ++l)
									{
										v[i][j][k][l] += c[m][l];
									}
								}
							}
						}
					}
				}
				for (l = 0; l < 3; l++)
					v[i][j][k][l] /= N;
			}
		}
	}
}
\end{lstlisting}

In the following section, we present the simulation of the flow around the spherical obstacle and round plate.

\bigskip

Particles with velocity in positive Z direction are created on a plain $z=1$, and at the plain $z = Z - 1$ they propagate freely out of the tunnel. 

The flow is actually happening in the tunnel with the square intersection. This tunnel is made of same 'material' as the obstacle, meaning that same no-slip condition holds at the tunnel and obstacle.

That no-slip condition holds at the obstacles can be seen from following excerpt of Propagation function (from FCHC, but same principle is applied in PI).
\begin{lstlisting}

/* Particle at the cell C[i] propagates along the lattice vector c[i] to a new node.  */
if (n & C[i])
{
	new_x = PeriodicBC(x + c[i][0], X);
	new_y = PeriodicBC(y + c[i][1], Y);
	new_z = PeriodicBC(z + c[i][2], Z);
	if (to[new_x][new_y][new_z] & OBS)	
	/* However, if there is obstacle in that node, it gets to the cell Reverse[i], that is cell diagonal to the C[i]. Let's say it had velocity v1 = [1,0,-1,0], by reflection it gained velocity v2 = [-1,0,1,0]. In the next step, particle propagates back to the node where it came from. */
	/* Hence, the velocity at the obstacle is v1 + v2 = 0, so we really fulfilled no-slip condition. */	
		to[new_x][new_y][new_z] |= Reverse[i];
	else
		to[new_x][new_y][new_z] |= C[i];
}
\end{lstlisting}

\section{Flow around the sphere} 
Each velocity vector is computed over $N = 80^3 / 8 = 64000$ nodes for PI and $N = 80^3 = 521 000$ for FCHC (by the function $compute_velocity$ that we presented).

\begin{figure}[htbp]
 \centering 
 \includegraphics[width=0.9\textwidth]{../PI_sphere_80/velocity_70}
 \label{transitions}
 \caption{PI - flow around sphere, time 70}
\end{figure}


\begin{figure}[htbp]
 \centering 
 \includegraphics[width=0.9\textwidth]{../PI_sphere_80/velocity_170}
 \label{transitions}
 \caption{PI - flow around sphere, time 170}
\end{figure}


\begin{figure}[htbp]
 \centering 
 \includegraphics[width=0.9\textwidth]{../PI_sphere_80/velocity_320}
 \label{transitions}
 \caption{PI - flow around sphere, time 320}
\end{figure}


\begin{figure}[htbp]
 \centering 
 \includegraphics[width=0.9\textwidth]{../PI_sphere_80/velocity_460}
 \label{transitions}
 \caption{PI - flow around sphere, time 460}
\end{figure}


\begin{figure}[htbp]
 \centering 
 \includegraphics[width=0.9\textwidth]{../PI_sphere_80/velocity_700}
 \label{transitions}
 \caption{PI - flow around sphere, time 700}
\end{figure}


\begin{figure}[htbp]
 \centering 
 \includegraphics[width=0.9\textwidth]{../PI_sphere_80/velocity_1000}
 \label{transitions}
 \caption{PI - flow around sphere, time 1000}
\end{figure}


\begin{figure}[htbp]
 \centering 
 \includegraphics[width=0.9\textwidth]{../PI_sphere_80/velocity_1500}
 \label{transitions}
 \caption{PI - flow around sphere, time 1500}
\end{figure}


\begin{figure}[htbp]
 \centering 
 \includegraphics[width=0.9\textwidth]{../PI_sphere_80/velocity_2300}
 \label{transitions}
 \caption{PI - flow around sphere, time 2300}
\end{figure}


\section{Flow around the disk}
Flow around the disk was performed on the Pair-interaction model only, the macroscopic velocity field was obtained by averaging over $N = 40^3 /8 = 8000$ nodes.

\begin{figure}[htbp]
 \centering 
 \includegraphics[width=0.9\textwidth]{../PI_plate_40/velocity_50}
 \label{transitions}
 \caption{PI - flow around plate, time 50}
\end{figure}

\begin{figure}[htbp]
 \centering 
 \includegraphics[width=0.9\textwidth]{../PI_plate_40/velocity_150}
 \label{transitions}
 \caption{PI - flow around plate, time 150}
\end{figure}

\begin{figure}[htbp]
 \centering 
 \includegraphics[width=0.9\textwidth]{../PI_plate_40/velocity_250}
 \label{transitions}
 \caption{PI - flow around plate, time 250}
\end{figure}


\begin{figure}[htbp]
 \centering 
 \includegraphics[width=0.9\textwidth]{../PI_plate_40/velocity_350}
 \label{transitions}
 \caption{PI - flow around plate, time 350}
\end{figure}

\begin{figure}[htbp]
 \centering 
 \includegraphics[width=0.9\textwidth]{../PI_plate_40/velocity_350}
 \label{transitions}
 \caption{PI - flow around plate, time 350}
\end{figure}

\begin{figure}[htbp]
 \centering 
 \includegraphics[width=0.9\textwidth]{../PI_plate_40/velocity_550}
 \label{transitions}
 \caption{PI - flow around plate, time 550}
\end{figure}

\begin{figure}[htbp]
 \centering 
 \includegraphics[width=0.9\textwidth]{../PI_plate_40/velocity_750}
 \label{transitions}
 \caption{PI - flow around plate, time 750}
\end{figure}

\begin{figure}[htbp]
 \centering 
 \includegraphics[width=0.9\textwidth]{../PI_plate_40/velocity_1000}
 \label{transitions}
 \caption{PI - flow around plate, time 1000}
\end{figure}

\begin{figure}[htbp]
 \centering 
 \includegraphics[width=0.9\textwidth]{../PI_plate_40/velocity_1500}
 \label{transitions}
 \caption{PI - flow around plate, time 1500}
\end{figure}

\begin{figure}[htbp]
 \centering 
 \includegraphics[width=0.9\textwidth]{../PI_plate_40/velocity_2000}
 \caption{PI - flow around plate, time 2000}
\end{figure}

\begin{figure}[htbp]
 \centering 
 \includegraphics[width=0.9\textwidth]{../PI_plate_40/velocity_2500}
 \caption{PI - flow around plate, time 2500}
\end{figure}
