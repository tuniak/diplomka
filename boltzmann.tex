\chapter{From Boltzmann to Navier-Stokes}

Because microdynamics of LGCA lead to equilibrium occupation numbers give by Fermi-Dirac distribution, there is no way to improve them substantionally u

In previous chapter we saw that microdynamics of LGCA lead to equilibrium occupation numbers give by Fermi-Dirac distribution.
Treatment of the symptoms resulting from this flaw is imperfect.

In this chapter we will show what statistical properties would be desirable for better computational model.
We will show that microdynamics of physical fluid lead to Boltzmann distribution, and starting there, we will derive Navier-Stokes equations in its most general form.
 
Indeed, the new generation of LGCA, the Lattice-Boltzmann models, were pushed one ladder of abstraction higher to meet the Boltzmann distribution of real fluids.
But since these models do not have discrete microdynamics anymore, they are not CA in its true sense, and they are beyond the scope of this thesis.
However, we would like to show some light at the tunnel and motivate the method, that is an efficient competitor to other CFD methods.

\bigskip
The name of the chapter is shamelessly stollen from \cite{wolf}. In the original piece, it precedes chapter on Lattice-Boltzman model and is full of technicalities (we have seen Chapman - Enskog expansion already), we rather follow approach of \cite{astro}, that offers better physical insight.

In Wolf-Gladrow, the derivation of Navier-Stokes is performed by the means of Chapman-Enskog expansion.
We chose approach from \cite{astro}, as it offers us better physical insight with less technicalities then \cite{wolf}, where the derivation of Navier-Stokes is performed by the means of Chapman-Enskog expansion.

\section{Boltzmann equation}

Let us consider an enslable of identical particles. Quantum effects, such as internal degrees of freedom in particles will be neglected. The state of the ensamble will be given by distribution function $f(\bm{x}, \bm{u}, t)$, that is defined by
\begin{equation} \label{hust}
dN = f(\bm{x}, \bm{u}, t) d\bm{x} \, d\bm{u}
\end{equation} 
where $dN$ is the number of particles in the infinitesimal phase element $[x, x+ dx] \cross [u, u +du]$.
It must be emphasized that statistical description is appropriate only if $dN$ is sufficiently large number.

Let us suppose that external force field $\bm{F} = (F_1, F_2, F_3)$ is affecting the system such that it is same for all particles in volume element $[x, x+ dx]$. This assumption is reasonable, since the particles are infinitesimally close.

Then, at any time $t$ the velocity of the particles in the element $[x, x + dx]$ will change from (let's say) $\bm{u'}$ to $\bm{u'} + \bm{F}dt$.

\begin{align*}
\pd_t f + \bm{v}.\nabla f + \frac{\bm{K}}{m}\pd_{\bm{v}} f = Q(f,f)
\end{align*}

Thus, if we neglect the collisions between particles, we can say, that number of particles 
$f(\bm{x}, \bm{u}, t)$ is equal to $f(\bm{x + u}dt, \bm{u + F}dt, t + dt)$, or 
\begin{align*}
f(\bm{x}, \bm{u}, t) - f(\bm{x + u}dt, \bm{u + F}dt, t + dt) = 0.
\end{align*}
If we consider the collisions between particles, we need to write
\begin{align*}
f(\bm{x}, \bm{u}, t) - f(\bm{x + u}dt, \bm{u + F}dt, t + dt) = [\Delta f]_{coll},
\end{align*}
where right hand side stands for change of $f$ in a time interval $dt$ due to collisions.

Since the left hand side is the difference of function values in infinitesimal time difference $dt$, we can write it in the differential form
\begin{align} \label{BoltzmannE}
\frac{\pd f}{\pd t} + u_i \frac{\pd f}{\pd x_i} + F_i \frac{\pd f}{\pd u_i} = \big[\frac{\pd f}{\pd t} \big]_{coll}.
\end{align}

This is the famous Boltzmann equation and we derived it in very intuitive way. We need to mention three subtle assumptions that Boltzmann used to derive it.
\begin{enumerate}
\item Only one-particle collisions are considered (so it would apply only to diluted gas).
\item \textit{Molecular chaos hypothesis} - velocities of colliding particles are uncorrelated.
\item External forces do not influence the dynamics of collisions.
\end{enumerate}

\section{Macroscopic quantities}
To obtain the volume density of particles $n(\bm{x},t)$, we integrate the probability density function $f(\bm{x},\bm{u},t)$ over velocity coordinates:
\begin{align*}
n(\bm{x},t) = \int f(\bm{x,u},t) \dd \bm{x} \dd \bm{u}.
\end{align*}

Then the mass density is
\begin{align} \label{densb}
\rho(\bm{x},t) = \int m \, f(\bm{x,u},t) \dd \bm{u},
\end{align}
and the mean velocity of the flow
\begin{align} \label{meanv}
\bm{v}(\bm{x},t) = \int \bm{u} \, f(\bm{x,u},t) \dd \bm{u}.
\end{align}

For the low densities we can neglect many-particle collisions and  consider only two-particle collisions that are elastic (preserving energy and momentum of the colliding pair).
Then, the density function $f$ is the famous Maxwell function
\begin{equation} \label{Maxwell}
f^M(\bm{x,v,}t) = n(\frac{m}{2 \pi k_B T})^{3/2} e^{-\frac{m}{2k_B T}(\bm{v-u})^2}.
\end{equation}

\section{Hydrodynamic equations}
The right hand side of \ref{boltzmanE} is often referred to as collision integral and is a lengthy expression that we do not to state explicitly, but its general properties can be analyzed nevertheless.

Since the collisions among the particles preserve the number of particles in the system, we have
\begin{align} \label{consm}
\int \Big[\frac{\pd f}{\pd t} \Big]_{coll} \dd \bm{u} = 0.
\end{align}
The total momentum is also preserved by the collisions, so
\begin{align} \label{consmm}
\int \bm{u} \Big[\frac{\pd f}{\pd t} \Big]_{coll} \dd \bm{u} = 0,
\end{align}
and finally, the total energy is conserved by the collisions, hence
\begin{align} \label{conse}
\int u^2 \Big[\frac{\pd f}{\pd t} \Big]_{coll} \dd \bm{u} = 0.
\end{align}

And because the total energy of the system is finite, it must be true that
\begin{align} \label{lime}
\lim_{u \rightarrow \infty} u^2 f = 0.
\end{align}

Using this limit and performing integration by parts on equations \ref{consm} -- \ref{conse} leads to
\begin{align} \label{due}
\int \frac{\pd f}{\pd u_i} \dd \bm{u} = 0,
\end{align}
\begin{align} \label{due2}
\int u_i \frac{\pd f}{\pd u_i} \dd \bm{u} = -\delta_{ij} \frac{\rho}{m},
\end{align}
\begin{align}
\frac{1}{2} \int u^2 \frac{\pd f}{\pd u_i} \dd \bm{u} = - v_i \frac{\rho}{m}.
\end{align}


Now that we are equipped with equations above, let us multiply the Boltzmann equation by $m$ and integrate over the whole domain. We obtain
\begin{align}
m \int \frac{\pd f}{\pd t} \dd \bm{u} + m \int u_i \frac{\pd f}{\pd x_i} \dd \bm{u} + m F_i \int \frac{\pd f}{\pd u_i} \dd \bm{u} = 0,
\end{align}
where right-hand side is zero due to equation \ref{consmm}.

We can delete the third term due to \ref{due}, and since $\bm{u}$ and $\bm{x}$ are considered to be independent variables in the phase space, we can swap the integration by $u_i$ and derivation by $x_i$ and $t$. Hence we get
\begin{align}
\frac{\pd }{\pd t} \int m \, f \dd \bm{u} + \frac{\pd}{\pd x_i} \int u_i \, m \, f \, \dd \bm{u} = 0,
\end{align}
and applying definition of density \ref{denb} and mean velocity \ref{meanv}
we can write in the form
\begin{align}
\frac{\pd \rho}{\pd t} + \frac{\pd}{\pd x_i}(\rho v_i) = 0.
\end{align}
This is the common form of continuity equation, or law of mass conservation.

If we multiply the Boltzmann equation by the mass $m$ and integrate over the whole domain again, we get
\begin{align} \label{primeul}
\frac{\pd}{\pd t} (\rho \, v_i) + \frac{\pd }{\pd x_j} \int m \, u_i \, u_j \, f \, \dd \bm{u} - \rho \, F_i = 0,
\end{align}
where we used equality \ref{due2}.

We can break down the second term into
\begin{align} 
\int m \, u_i \, u_j \, f \, \dd \bm{u} = \int m \, v_i \, v_j \, f \dd \bm{u} + \int (v_i - u_i) (v_j - u_j) \, f \, \dd \bm{u} = \rho \, v_i \, v_j + P_{ij}
\end{align}
where we define 
\begin{align}
P_{ij} = \int m (v_i - u_i) (v_j - u_j) \, f \, \dd \bm{u}.
\end{align}
In case that pressure is isotropic, it is proportional to Kronecker delta
\begin{align}
P_{ij} = P \, \delta_{ij}
\end{align}
with 
\begin{align}
P = \frac{1}{3}P_{ii}.
\end{align}
Hence equation \ref{primeul} simplifies to
\begin{align}
\frac{\pd}{\pd t} (\rho \, v_i) + \frac{\pd }{\pd x_j} (\rho v_i v_j) =  - \frac{\pd P}{\pd x_i}+ \rho \, F_i
\end{align}

or in the vector form
\begin{align}
\frac{\pd }{\pd t} (\rho \bm{v}) + \nabla . \pi = \rho \bm{F}
\end{align}
where
\begin{align}
\pi_{ij} = \rho \, v_i \, v_j + P \delta_{ij}
\end{align}
This is the second hydrodynamic equation, expressing the conservation of momentum.
