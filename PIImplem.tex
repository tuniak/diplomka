\chapter{Implementation of the Pair-Interaction LGCA in 3D}


\begin{chapquote}{Donald Knuth, \textit{Structured programming with GoTo statement}}
"Programmers waste enormous amounts of time thinking about, or worrying about, the speed of noncritical parts of their programs, and these attempts at efficiency actually have a strong negative impact when debugging and maintenance are considered. We should forget about small efficiencies, say about 97\% of the time: premature optimization is the root of all evil. Yet we should not pass up our opportunities in that critical 3\%."
\end{chapquote}
%
%\epigraph{Programmers waste enormous amounts of time thinking about, or worrying about, the speed of noncritical parts of their programs, and these attempts at efficiency actually have a strong negative impact when debugging and maintenance are considered. We should forget about small efficiencies, say about 97\% of the time: premature optimization is the root of all evil. Yet we should not pass up our opportunities in that critical 3\%.}{Donald Knuth in "Structured programming with GoTo statement"}


%We followed best practices of in \cite{},
Inspired by this quote and best programming practice presented in \cite{lbm} on Lattice-Boltzmann implementation, we did our best to implement functions \textit{collision} and \textit{propagations} efficiently, since they are the core of the simulation and overall performance depends largely on them.

Excerpts of the source code that we present are already parallelized by OpenMP.

\section{Implementation of the collision algorithm}
\begin{lstlisting}
#define A 1
#define B 2
#define C 4
#define D 8
#define E 16
#define F 32
#define G 64
#define H 128

unsigned char cell[8] = {A, B, C, D, E, F, G, H};


/* Node consists of 8 mass bits (char m) and 3x8 momentum bits (char p[3]). If there is an obstacle in the node <=>  node.o == 1. */
typedef struct
{
	unsigned char m;
	unsigned char p[3];
	int o;
} Node;
// We access the bits of the of the node by bitwise AND opperation
//e.g. node.m & A == 1 means there is particle in the cell A
// 		node.p[1] & A == 0 means that this particle has no momentum in Y-direction

// 3 directions (X, Y, Z), 4 pair-interactions in each direction, 2 cells in each pair 
unsigned char Pair[3][4][2] = 
{
	{
		{A,E}, {B,F}, {C,G}, {D,H}
	},
	{
		{A,C}, {B,D}, {E,G}, {F,H}
	},
	{
		{A,B}, {C,D}, {E,F}, {G,H}
	}
};

// The collision algorithm
void collision(Node &node)
{
	//d,u ... index for downer and upper momentum
	int d; //index of momentum in direction of PI
	int u; //index of momentum orthogonal to direction of PI

	// l, r ... left/right cell in the pair
	// ml, mr ... mass bits in the pair
	// lu, ld, ri, ru, rd, ri ... momenta bits in the pair
	unsigned char l, r, ml, mr, lu, ld, ru, rd, li, ri;

	for (int i = 0; i < 3; ++i)
	{
		d = (i + 1) % 3;
		u = (i + 2) % 3;
		for (int j = 0; j < 4; ++j)
		{
			l = Pair[i][j][0];
			r = Pair[i][j][1];

			ml = node.m&l;
			mr = node.m&r;
			ld = node.p[d] & l;
			lu = node.p[u] & l;
			li = node.p[i] & l;
			rd = node.p[d] & r;
			ru = node.p[u] & r;
			ri = node.p[i] & r;

			/* PAIR INTERACTIONS */
			if (!ml && !mr)
				continue;
			if (ml && mr)
			{
				// alternate momenta in direction of the pair-interaction
				if (!ld && !rd)
				{
					node.p[d] |= l;
					node.p[d] |= r;
				}
				else if (ld && rd)
				{
					node.p[d] ^= l;
					node.p[d] ^= r;
				}
				// alternate momenta in all other directions
				if (lu && !ru)
				{
					node.p[u] ^= l;
					node.p[u] |= r;
				}
				else if (!lu && ru)
				{
					node.p[u] |= l;
					node.p[u] ^= r;
				}
				if (li && !ri)
				{
					node.p[i] ^= l;
					node.p[i] |= r;
				}
				else if (!li && ri)
				{
					node.p[i] |= l;
					node.p[i] ^= r;
				}
			}
			else if (!ml && !rd)
			{
				node.m |= l;
				node.m ^= r;
				if (ru)
				{
					node.p[u] |= l;
					node.p[u] ^= r;
				}
				if (ri)
				{
					node.p[i] |= l;
					node.p[i] ^= r;
				}					
			}
			else if (!mr && !ld)
			{
				node.m |= r;
				node.m ^= l;
				if (lu)
				{
					node.p[u] |= r;
					node.p[u] ^= l;
				}
				if (li)
				{
					node.p[i] |= r;
					node.p[i] ^= l;
				}
			}
		}
	}
}

// 'Collision' runs over the lattice and calls function 'collision' on each node. 
void Collision(Node***array, int X, int Y, int Z, int start)
{
	int x,y,z;

#pragma omp parallel for private (x,y,z)
	for (x = start; x < X; x+=2)
	{
		for(y = start; y < Y; y+=2)
		{
			for(z = start; z < Z; z+=2)
			{
				// if the node is obstacle, skip it
				if(array[x][y][z].o)
					continue;
				collision(array[x][y][z]);
			}
		}
	}
}
\end{lstlisting}


\section{Implementation of the Propagation}
%Before we comment on the implementation, we want to emphasize one of the strong sides of PI automaton.

%In FCHC, the lattice consisted of all the nodes with the discrete Cartesian coordinates (up to some upper boundary denoted by X, Y, Z).
%All of these nodes were interconnected and formed single lattice.

As we explained in the teoretical part, the PI-3D lattice splits into 4 sub-lattices, that are not connected. We are using only one of these sub-lattices. At even times $t=2,4,...$ only nodes with even indices are populated, and at odd times, nodes with odd indices are populated.

Theoretically, this setting guarantees that spurious Zannetti's invariants are not present (\cite{nasilowski}), practically, we can use single data structure to store the lattice before and after propagation. Comparing to other lattices (FHP, FCHC, DdQq), this provides additional speed-up by lowering memory bandwidth and saving other resources.

The excerpts of the source code governing Propagation follows. 
\begin{lstlisting}
/* Group of cells that propagates in positive X-direction, Y-direction and Z-direction. */
#define dirX (B+D+F+H)
#define dirY (E+F+G+H)
#define dirZ (A+B+E+F)
/* For example, A & dirX == 0, so particle from A propagates in negative X direction,
    but A & dirZ == 1, so it propagates in positive Z direction. */

/* At start == 0, particles are streamed from even nodes ([0,2,124]...) to odd nodes([1,1,123]...) 
   At start == 1, from odd to even nodes. */
void Propagation(Node***array, int X, int Y, int Z, int start)
{

#pragma omp parallel for
	for (int x = start; x < X; x += 2)
	{
		for (int y = start; y < Y; y += 2)
		{
			for (int z = start; z < Z; z += 2)
			{
				// for cells in the node
				for (int k = 1; k <= H; k <<= 1)
				{
					//if there is particle in the cell, we propagate it
					if (array[x][y][z].m & k)
					{
						int xN = k & dirX ? (x + 1) % X : (x - 1 + X) % X;
						int yN = k & dirY ? (y + 1) % Y : (y - 1 + Y) % Y;
						int zN = k & dirZ ? (z + 1) % Z : (z - 1 + Z) % Z;

#pragma omp atomic
						array[xN][yN][zN].m |= k & array[x][y][z].m;
#pragma omp atomic
						array[xN][yN][zN].p[0] |= k & array[x][y][z].p[0];
#pragma omp atomic
						array[xN][yN][zN].p[1] |= k & array[x][y][z].p[1];
#pragma omp atomic
						array[xN][yN][zN].p[2] |= k & array[x][y][z].p[2];
					}
				}
				//we set the old node to 0
				array[x][y][z].m = 0;
				array[x][y][z].p[0] = 0;
				array[x][y][z].p[1] = 0;
				array[x][y][z].p[2] = 0;
			}
		}
	}
}
\end{lstlisting}